%%
%% KTH Master's Thesis: FBMC vs ATC for Nordic Electricity Market
%% Migrated to new KTH template
%%

\RequirePackage{ifxetex}
\RequirePackage{ifluatex}
\newif\ifxeorlua
\ifxetex\xeorluatrue\fi
\ifluatex\xeorluatrue\fi

\ifxeorlua
\RequirePackage{expl3}
\RequirePackage{etoolbox}
\ExplSyntaxOn
\pdf_version_gset:n{1.5}
\ExplSyntaxOff
\else
\RequirePackage{expl3}
\ExplSyntaxOn
\pdf_version_gset:n{1.5}
\ExplSyntaxOff
\fi

\makeatletter
\newcommand{\disablepackage}[2]{%
  \disable@package@load{#1}{#2}%
}
\newcommand{\reenablepackage}[1]{%
  \reenable@package@load{#1}%
}
\makeatother
\ifxeorlua
\disablepackage{transparent}{}
\fi

\documentclass[
english,
bibtex,
]{kththesis}

\iftoggle{biblatex}{
    \usepackage[style=ieee,citestyle=numeric-comp]{biblatex}
    \addbibresource{references.bib}
}{
    \bibliographystyle{bibstyle/myIEEEtran}
}

\input{lib/includes}
\input{lib/kthcolors}
\input{lib/defines}

\makeatletter
\let\verbatimsc\@undefined
\let\endverbatimsc\@undefined
\lst@AddToHook{Init}{\hyphenpenalty=50\relax}
\makeatother

\lstnewenvironment{verbatimsc}
    {
    \lstset{%
        basicstyle=\ttfamily\tiny,
        backgroundcolor=\color{white},
        columns=[l]fixed,
        language=[LaTeX]TeX,
        keywordstyle=\color{red},
        breaklines=true,
        breakatwhitespace=true,
        breakindent=0em,
        frame=none,
        postbreak={}
    }
}{}

\newcolumntype{L}[1]{>{\raggedright\let\newline\\\arraybackslash\hspace{0pt}}p{#1}}

\iftoggle{biblatex}{
    \usepackage[plainpages=false]{hyperref}
}{
    \usepackage[
    backref=page,
    pagebackref=false,
    plainpages=false,
    unicode=true,
    bookmarks=true,
    bookmarksopen=false,
    pdfpagemode=UseNone,
    destlabel,
    pdfencoding=auto,
    ]{hyperref}
}


\usepackage[all]{hypcap}

\usepackage[acronym, style=super, section=section, nonumberlist, nomain,
nopostdot]{glossaries}
\setlength{\glsdescwidth}{0.75\textwidth}
\usepackage[]{glossaries-extra}

\newglossary[tlg]{readme}{tld}{tdn}{README acronyms}

\input{lib/includes-after-hyperref}

\makeglossaries

\ifxeorlua
%%% Local Variables:
%%% mode: latex
%%% TeX-master: t
%%% End:
% The following command is used with glossaries-extra
\setabbreviationstyle[acronym]{long-short}
% The form of the entries in this file is \newacronym{label}{acronym}{phrase}
%                                      or \newacronym[options]{label}{acronym}{phrase}
% see "User Manual for glossaries.sty" for the  details about the options, one example is shown below
% note the specification of the long form plural in the line below
\newacronym[longplural={Debugging Information Entities}]{DIE}{DIE}{Debugging Information Entity}
%
% The following example also uses options
\newacronym[shortplural={OSes}, firstplural={operating systems (OSes)}]{OS}{OS}{operating system}

% note the use of a non-breaking dash in long text for the following acronym
\newacronym{IQL}{IQL}{Independent Q‑Learning}

% example of putting in a trademark on first expansion
\newacronym[first={NVIDIA OpenSHMEM Library (NVSHMEM\texttrademark)}]{NVSHMEM}{NVSHMEM}{NVIDIA OpenSHMEM Library}

\newacronym{API}{API}{Application Programming Interface}
\newacronym{DiVA}{DiVA}{Digitala Vetenskapliga Arkivet}
\newacronym{IMRAD}{IMRAD}{Introduction, Methods, Results, and Discussione}
\newacronym{JSON}{JSON}{JavaScript Object Notation}
\newacronym{KI}{KI}{Karolinska Institutet}
\newacronym{KOPPS}{KOPPS}{Kurs- och programplaneringssystemet}
\newacronym{LADOK}{LADOK}{Lokalt adb–baserat dokumentationssystemt}
\newacronym{TIMTM}{TIMTM}{Interactive Media Technology}
\newacronym{TMMTM}{TMMTM}{Media Managemen}
\newacronym{KTH}{KTH}{KTH Royal Institute of Technology}

\newacronym{LAN}{LAN}{Local Area Network}
\newacronym{VM}{VM}{virtual machine}
% note the use of a non-breaking dash in the following acronym
\newacronym{WiFi}{Wi‑Fi}{Wireless Fidelity}

\newacronym{WLAN}{WLAN}{Wireless Local Area Network}
\newacronym{UN}{UN}{United Nations}
\newacronym{SDG}{SDG}{Sustainable Development Goal}

% FBMC vs ATC Thesis Acronyms
% Core methodology
\newacronym{FBMC}{FBMC}{Flow-Based Market Coupling}
\newacronym{ATC}{ATC}{Available Transfer Capacity}
\newacronym{NTC}{NTC}{Net Transfer Capacity}
\newacronym{PTDF}{PTDF}{Power Transfer Distribution Factor}
\newacronym{LODF}{LODF}{Line Outage Distribution Factor}
\newacronym{CNEC}{CNEC}{Critical Network Element with Contingency}
\newacronym{RAM}{RAM}{Remaining Available Margin}
\newacronym{GSK}{GSK}{Generation Shift Key}

% Organizations and regulations
\newacronym{CACM}{CACM}{Capacity Allocation and Congestion Management}
\newacronym[longplural={European Network of Transmission System Operators for Electricity}]{ENTSO-E}{ENTSO-E}{European Network of Transmission System Operators for Electricity}
\newacronym[shortplural={TSOs}]{TSO}{TSO}{Transmission System Operator}
\newacronym{RCC}{RCC}{Regional Coordination Centre}
\newacronym{ACER}{ACER}{Agency for the Cooperation of Energy Regulators}

% Regions
\newacronym{CWE}{CWE}{Central Western Europe}
\newacronym{EU}{EU}{European Union}

% Technical
\newacronym{DC}{DC}{Direct Current}
\newacronym{AC}{AC}{Alternating Current}
\newacronym{OPF}{OPF}{Optimal Power Flow}
\newacronym{LP}{LP}{Linear Programming}

\else
\input{lib/acronyms-for-pdflatex}
\fi

\input{lib/placeHolder_lbx_files}
\input{custom_configuration}
\IfFileExists{custom_configuration_plaintext.tex}{\input{custom_configuration_plaintext.tex}}{}

\EnglishKeywords{Flow-Based Market Coupling, Available Transfer Capacity, Nordic Electricity Market, Power Transfer Distribution Factors, Congestion Management, Zonal Pricing, Market Efficiency}
\SwedishKeywords{Flödesbaserad marknadskoppling, Tillgänglig överföringskapacitet, Nordisk elmarknad, Effektöverföringsfaktorer, Trängselhantering, Zonprissättning, Marknadseffektivitet}

\presentationDateAndTimeISO{2025-06-15 10:00}
\presentationLanguage{eng}
\presentationRoom{TBD}
\presentationAddress{TBD}
\presentationCity{Stockholm}
\opponentsNames{TBD}

% TODO: Update TRITA number after thesis approval (assigned by student affairs office)
\trita{TRITA -- EECS-EX}{2025:0000}

\input{lib/pdf_related_includes}

\hypersetup{
	colorlinks  = true,
	breaklinks  = true,
	linkcolor   = black,
	urlcolor    = black,
	citecolor   = black,
	anchorcolor = black
}

\ifnomenclature
\renewcommand*{\pagedeclaration}[1]{\unskip, \dotfill\hyperlink{page.#1}{page\nobreakspace#1}}
\renewcommand{\nomname}{List of Symbols Used}
\renewcommand{\nompreamble}{The following symbols will be later used within the body of the thesis.}
\makenomenclature
\fi

\usepackage{subfiles}

\listfiles

\begin{document}
\selectlanguage{english}

\pagenumbering{alph}
\kthcover
\clearpage\thispagestyle{empty}\mbox{}
\titlepage
\bookinfopage

\frontmatter
\setcounter{page}{1}

%%%%%%%%%%%%%%%%%%%%%%%%%%%%%%%%%%%%%%%%%%%%%%%%%%%%%%%%%%%%%%%%%%
%% ENGLISH ABSTRACT
%%%%%%%%%%%%%%%%%%%%%%%%%%%%%%%%%%%%%%%%%%%%%%%%%%%%%%%%%%%%%%%%%%
\begin{abstract}
\markboth{\abstractname}{}
\begin{scontents}[store-env=lang]
eng
\end{scontents}

\begin{scontents}[store-env=abstracts,print-env=true]
The European electricity market is undergoing a fundamental transformation in cross-border transmission capacity allocation, with Flow-Based Market Coupling (FBMC) replacing the traditional Available Transfer Capacity (ATC) methodology. This thesis presents a quantitative comparison of FBMC and ATC for the Nordic electricity market, focusing on the Swedish transmission system following the October 2024 FBMC implementation.

A 13-node zonal market model was developed representing the four Swedish bidding zones (SE1--SE4) and nine neighboring market areas. The model implements DC power flow-based Power Transfer Distribution Factor (PTDF) calculations and linear programming optimization using Python with the HiGHS solver. Market clearing was simulated under both FBMC (PTDF-constrained) and ATC (NTC-limited) regimes using data from the ENTSO-E Transparency Platform and Svenska kraftnät.

The results demonstrate that FBMC produces measurable efficiency gains over ATC, but with significant seasonal variation. During winter conditions (December 2024), FBMC achieved 6.5\% cost savings (4.37 million EUR weekly) compared to ATC, driven by high congestion on the SE2--SE3 corridor (88\% utilization). In contrast, summer conditions (July 2024) showed only 0.25\% savings due to reduced congestion (53\% utilization), representing a 50-fold seasonal differential.

Sensitivity analysis across five scenarios confirmed that FBMC consistently outperforms ATC, though benefit magnitude varies substantially from 0\% to 16.1\% depending on congestion levels and price conditions. Two findings validate the underlying mechanism: FBMC benefits amplify under constrained transmission (16.1\% savings with 30\% SE2–SE3 capacity reduction) and disappear entirely under price convergence conditions, confirming that the mechanism operates through enhanced arbitrage facilitation. The PTDF matrix analysis reveals non-trivial flow distribution factors for the SE2–NO4 corridor (0.286), indicating that cross-border connections create alternative routing paths through Norwegian interconnections. This suggests FBMC benefits arise from loop flow exploitation rather than improvements to Sweden's internal radial corridors, though this topology dependency was not independently tested through corridor isolation scenarios.

These findings have implications for Nordic transmission planning and demonstrate that FBMC delivers greatest value during high-congestion periods when transmission constraints actively bind.
\end{scontents}

\subsection*{Keywords}
\begin{scontents}[store-env=keywords,print-env=true]
\InsertKeywords{english}
\end{scontents}
\end{abstract}
\cleardoublepage

%%%%%%%%%%%%%%%%%%%%%%%%%%%%%%%%%%%%%%%%%%%%%%%%%%%%%%%%%%%%%%%%%%
%% SWEDISH ABSTRACT (Sammanfattning)
%%%%%%%%%%%%%%%%%%%%%%%%%%%%%%%%%%%%%%%%%%%%%%%%%%%%%%%%%%%%%%%%%%
\babelpolyLangStart{swedish}
\begin{abstract}
\markboth{\abstractname}{}
\begin{scontents}[store-env=lang]
swe
\end{scontents}

\begin{scontents}[store-env=abstracts,print-env=true]
Den europeiska elmarknaden genomgår en fundamental förändring i hur gränsöverskridande överföringskapacitet allokeras, där flödesbaserad marknadskoppling (FBMC) ersätter den traditionella metoden med tillgänglig överföringskapacitet (ATC). Detta examensarbete presenterar en kvantitativ jämförelse mellan FBMC och ATC för den nordiska elmarknaden, med fokus på det svenska transmissionssystemet efter FBMC-implementeringen i oktober 2024.

En 13-nodsmodell utvecklades för att representera de fyra svenska elområdena (SE1--SE4) samt nio angränsande marknadsområden. Modellen implementerar DC-lastflödesbaserade beräkningar av effektöverföringsfaktorer (PTDF) och linjärprogrammeringsoptimering med Python och HiGHS-lösaren. Marknadsklarering simulerades under både FBMC (PTDF-begränsad) och ATC (NTC-begränsad) med data från ENTSO-E Transparency Platform och Svenska kraftnät.

Resultaten visar att FBMC ger mätbara effektivitetsvinster jämfört med ATC, men med betydande säsongsvariation. Under vinterförhållanden (december 2024) uppnådde FBMC 6,5\% kostnadsbesparingar (4,37 miljoner EUR per vecka) jämfört med ATC, drivet av hög trängsel på SE2--SE3-korridoren (88\% utnyttjande). Under sommarförhållanden (juli 2024) var besparingarna endast 0,25\% på grund av minskad trängsel (53\% utnyttjande), vilket representerar en 50-faldig säsongsskillnad.

Känslighetsanalys över fem scenarier bekräftade robustheten i dessa resultat: FBMC-fördelarna ökar vid begränsad transmission (16,1\% besparing vid 30\% SE2--SE3-kapacitetsreduktion) och försvinner helt vid priskonvergens, vilket validerar att mekanismen fungerar genom förbättrad arbitragefacilitering. Dessa resultat visar att FBMC levererar störst värde under perioder med hög trängsel när transmissionsbegränsningar aktivt binder.
\end{scontents}

\subsection*{Nyckelord}
\begin{scontents}[store-env=keywords,print-env=true]
\InsertKeywords{swedish}
\end{scontents}
\end{abstract}
\cleardoublepage
\babelpolyLangStop{swedish}

%%%%%%%%%%%%%%%%%%%%%%%%%%%%%%%%%%%%%%%%%%%%%%%%%%%%%%%%%%%%%%%%%%
%% ACKNOWLEDGMENTS
%%%%%%%%%%%%%%%%%%%%%%%%%%%%%%%%%%%%%%%%%%%%%%%%%%%%%%%%%%%%%%%%%%
\section*{Acknowledgments}
\markboth{Acknowledgments}{}

I would like to express my sincere gratitude to my supervisor, Saeed Mohammadi, and my examiner, Professor Mohammad Reza Hesamzadeh their guidance, expertise and support throughout this project. 

%I am grateful to Svenska kraftnät and ENTSO-E for making comprehensive market data publicly available through their transparency platforms, enabling independent research on European electricity markets.

%Finally, I thank my family and friends for their support during my studies at KTH.

\acknowlegmentssignature

\fancypagestyle{plain}{}
\renewcommand{\chaptermark}[1]{ \markboth{#1}{}}
\tableofcontents
\markboth{\contentsname}{}

\cleardoublepage
\listoffigures

\cleardoublepage
\listoftables

\cleardoublepage
\newglossarystyle{mylong}{%
  \setglossarystyle{long}%
  \renewenvironment{theglossary}%
     {\begin{longtable}[l]{@{}p{\dimexpr 2cm-\tabcolsep}p{0.8\hsize}}}%
     {\end{longtable}}%
 }
\printglossary[style=mylong, type=\acronymtype, title={List of acronyms and abbreviations}]

\ifnomenclature
    \cleardoublepage
    \printnomenclature
\fi

\label{pg:lastPageofPreface}

%%%%%%%%%%%%%%%%%%%%%%%%%%%%%%%%%%%%%%%%%%%%%%%%%%%%%%%%%%%%%%%%%%
%% MAIN MATTER
%%%%%%%%%%%%%%%%%%%%%%%%%%%%%%%%%%%%%%%%%%%%%%%%%%%%%%%%%%%%%%%%%%
\mainmatter
\glsresetall
\renewcommand{\chaptermark}[1]{\markboth{#1}{}}
\selectlanguage{english}

%%%%%%%%%%%%%%%%%%%%%%%%%%%%%%%%%%%%%%%%%%%%%%%%%%%%%%%%%%%%%%%%%%
%% CHAPTER 1: INTRODUCTION
%%%%%%%%%%%%%%%%%%%%%%%%%%%%%%%%%%%%%%%%%%%%%%%%%%%%%%%%%%%%%%%%%%
\chapter{Introduction}
\label{ch:introduction}

The efficient allocation of cross-border transmission capacity is fundamental to the operation of interconnected electricity markets. As European power systems become increasingly integrated, the methodology used for capacity allocation directly impacts market efficiency, price formation, and ultimately consumer welfare. This chapter introduces the context, problem statement, and objectives of this thesis.

\section{Background}
\label{sec:background}

The European Union's internal electricity market represents one of the world's largest integrated power systems, spanning multiple synchronous areas and dozens of bidding zones. The efficient operation of this market depends critically on how transmission capacity between bidding zones is allocated to market participants.

Historically, cross-border capacity allocation in European electricity markets has relied on the \gls{ATC} methodology. Under ATC, each transmission corridor is assigned a maximum transfer capacity value, the \gls{NTC}, which represents the maximum power that can be exchanged between two adjacent zones while respecting security constraints. These NTC values are determined by \glspl{TSO} based on worst-case assumptions about network conditions and are typically conservative to ensure N-1 security under all anticipated operating conditions~\cite{eu_cacm_2015}.

The fundamental limitation of ATC lies in its treatment of the physical network. By reducing the complex, meshed transmission grid to simple bilateral capacity limits, ATC cannot capture the physical reality that power flows distribute across multiple parallel paths according to Kirchhoff's laws. In meshed networks, a transaction between zones A and B will cause flows not only on the direct A-B corridor but also on alternative paths through intermediate zones---so-called loop flows. The ATC methodology must embed conservative margins to account for these effects, resulting in systematic underutilization of actual physical capacity.

\gls{FBMC} addresses this limitation by explicitly modeling the physical network through \gls{PTDF} matrices. PTDFs quantify how a unit injection at one node and withdrawal at another distributes across all network elements. By incorporating these factors into the market clearing algorithm, FBMC can identify feasible dispatch solutions that would be rejected under ATC's simplified constraints, while still respecting actual physical limits on each \gls{CNEC}.

The \gls{CWE} region pioneered FBMC implementation in May 2015, with subsequent analyses documenting welfare gains of 150--300 million EUR annually~\cite{acer_fbmc_2019}. Building on this experience, the Nordic electricity market---comprising Sweden, Norway, Finland, and Denmark---completed its transition to FBMC in October 2024 following an 18-month parallel operation period coordinated by the Nordic \gls{RCC}.

\section{Problem Definition}
\label{sec:problem}

The Swedish transmission system presents a compelling case study for analyzing FBMC effectiveness. Sweden's four bidding zones (SE1--SE4) exhibit a predominantly north-south radial topology, with the critical SE2--SE3 corridor representing the primary bottleneck constraining power transfer from hydroelectric-dominated northern regions to load-intensive southern consumption centers. This structural characteristic raises fundamental questions regarding the magnitude of efficiency gains achievable through FBMC in systems with limited internal loop flow opportunities.

\subsection{Research Questions}

The primary research question addressed by this thesis is:

\textbf{What is the quantitative difference in market efficiency between \gls{FBMC} and \gls{ATC} methods for the Swedish electricity market, and how does this difference vary with seasonal demand patterns and network congestion levels?}

This overarching question is decomposed into the following specific research objectives:

\begin{enumerate}
    \item \textbf{RQ1 (Efficiency Differential):} What is the magnitude of cost savings achieved by FBMC compared to ATC under representative operating conditions?

    \item \textbf{RQ2 (Topology Dependency):} How does the relative performance of FBMC versus ATC depend on network topology, specifically comparing radial internal transmission with meshed cross-border interconnections?

    \item \textbf{RQ3 (Seasonal Variation):} How do seasonal variations in demand patterns and congestion levels influence the distribution of FBMC benefits throughout the year?

    \item \textbf{RQ4 (Robustness):} How robust are the observed FBMC benefits to variations in exogenous parameters including neighboring zone prices and transmission capacities?
\end{enumerate}

\section{Purpose}
\label{sec:purpose}

The purpose of this thesis is twofold. First, from an academic perspective, it contributes to the understanding of capacity allocation mechanisms by providing quantitative evidence on FBMC performance in a system with distinct topological characteristics compared to the well-studied CWE region. Second, from a practical perspective, the findings inform policy discussions regarding Nordic market design and transmission investment planning.

The beneficiaries of this work include:
\begin{itemize}
    \item Nordic TSOs evaluating FBMC implementation outcomes
    \item Regulatory authorities assessing market efficiency
    \item Market participants developing trading strategies
    \item Academic researchers studying European electricity market integration
\end{itemize}

\section{Goals}
\label{sec:goals}

The concrete goals of this degree project are:

\begin{enumerate}
    \item \textbf{Model Development:} Develop a validated zonal market model capable of simulating both FBMC and ATC capacity allocation for the Nordic region

    \item \textbf{Quantitative Analysis:} Quantify the efficiency differential between FBMC and ATC across multiple time periods and scenarios

    \item \textbf{Mechanism Identification:} Identify the physical and economic mechanisms driving observed differences

    \item \textbf{Policy Implications:} Derive actionable insights for transmission planning and market design
\end{enumerate}

\section{Research Methodology}
\label{sec:methodology_intro}

This thesis employs quantitative simulation methodology combined with counterfactual analysis. A zonal market model is developed and validated against historical data, then used to simulate market outcomes under both FBMC and ATC regimes while holding all other factors constant. The difference in outcomes represents the causal effect of the capacity allocation methodology.

The choice of simulation methodology is motivated by several factors:
\begin{itemize}
    \item Controlled comparison: Real-world data cannot provide clean counterfactuals since only one methodology is operational at any time
    \item Parameter variation: Simulation enables systematic sensitivity analysis
    \item Reproducibility: Open-source tools and public data ensure results can be verified
\end{itemize}

Alternative methodologies considered but rejected include econometric analysis of before/after FBMC implementation (confounded by other time-varying factors) and agent-based modeling (added complexity without proportional insight for this research question).

\section{Delimitations}
\label{sec:delimitations}

This thesis is explicitly bounded by the following delimitations:

\begin{itemize}
    \item \textbf{Geographic scope:} Analysis focuses on the Swedish bidding zones with neighboring areas modeled as price-taking zones
    \item \textbf{Temporal scope:} Representative weeks from 2024 are analyzed; continuous annual simulation is not performed
    \item \textbf{Market scope:} Only the day-ahead market is modeled; intraday and balancing markets are excluded
    \item \textbf{Technical scope:} DC power flow approximation is used; AC power flow effects are not captured
    \item \textbf{Strategic behavior:} Market participants are assumed to bid at marginal cost; strategic bidding is not modeled
\end{itemize}

\section{Thesis Structure}
\label{sec:structure}

The remainder of this thesis is organized as follows:

Chapter~\ref{ch:background} provides theoretical background on electricity market design, FBMC methodology, and reviews related work on capacity allocation analysis.

Chapter~\ref{ch:methods} describes the research methodology, including the mathematical formulation, data sources, and experimental design.

Chapter~\ref{ch:implementation} details the model implementation, including PTDF calculation and optimization setup.

Chapter~\ref{ch:results} presents the results of base case, sensitivity, and multi-period analyses.

Chapter~\ref{ch:discussion} interprets the results and discusses their implications.

Chapter~\ref{ch:conclusions} summarizes conclusions, limitations, and directions for future work.


%%%%%%%%%%%%%%%%%%%%%%%%%%%%%%%%%%%%%%%%%%%%%%%%%%%%%%%%%%%%%%%%%%
%% CHAPTER 2: BACKGROUND
%%%%%%%%%%%%%%%%%%%%%%%%%%%%%%%%%%%%%%%%%%%%%%%%%%%%%%%%%%%%%%%%%%
\cleardoublepage
\chapter{Background}
\label{ch:background}

This chapter provides the theoretical foundation for understanding capacity allocation in electricity markets. Section~\ref{sec:electricity_markets} introduces electricity market fundamentals. Section~\ref{sec:atc_theory} describes the ATC methodology. Section~\ref{sec:fbmc_theory} presents FBMC theory. Section~\ref{sec:nordic_market} describes the Nordic electricity market. Section~\ref{sec:related_work} reviews related academic work.

\section{Electricity Market Fundamentals}
\label{sec:electricity_markets}

Modern electricity markets operate through a sequence of trading platforms with different time horizons. The day-ahead market, operated by power exchanges such as Nord Pool in the Nordic region, determines hourly prices and quantities for the following day through a uniform-price auction. Market participants submit bids specifying the quantity they wish to buy or sell at various price levels, and the market operator determines the market-clearing price where aggregate supply equals aggregate demand.

In zonal pricing systems, the market is divided into bidding zones within which a single price applies. When transmission capacity between zones is sufficient to accommodate all desired trades, prices converge across zones. When transmission constraints bind, zones decouple and different prices emerge, reflecting the marginal cost of congestion.

\section{Available Transfer Capacity Method}
\label{sec:atc_theory}

Under the ATC methodology, each interconnector between bidding zones is assigned a \gls{NTC} value representing the maximum commercial exchange permitted. The market clearing problem can be formulated as:

\begin{equation}
\min_{g, f} \sum_{n} C_n \cdot g_n
\end{equation}

subject to:
\begin{align}
\sum_{g \in G_n} g - D_n + \sum_{l: to(l)=n} f_l - \sum_{l: from(l)=n} f_l &= 0 \quad \forall n \\
|f_l| &\leq NTC_l \quad \forall l \\
g_n^{min} &\leq g_n \leq g_n^{max} \quad \forall n
\end{align}

where $g_n$ is generation at node $n$, $C_n$ is marginal cost, $D_n$ is demand, and $f_l$ is flow on corridor $l$.

The key limitation is that NTC values must be set conservatively to ensure security under all anticipated flow patterns, since the ATC formulation cannot distinguish between different flow distributions that result in the same net exchange.

\section{Flow-Based Market Coupling}
\label{sec:fbmc_theory}

FBMC replaces the simplified NTC constraints with explicit representation of power flow physics through \gls{PTDF} matrices. The PTDF coefficient $PTDF_{l,n}$ quantifies the fraction of a unit injection at node $n$ (with corresponding withdrawal at the slack bus) that flows through line $l$.

For a DC power flow approximation, PTDFs are calculated as:
\begin{equation}
PTDF = B_f \cdot B^{-1}
\end{equation}

where $B_f$ is the branch susceptance matrix and $B$ is the nodal susceptance matrix.

The FBMC market clearing problem becomes:
\begin{equation}
\min_{g, p} \sum_{n} C_n \cdot g_n
\end{equation}

subject to:
\begin{align}
g_n - D_n &= p_n \quad \forall n \\
\left| \sum_{n} PTDF_{l,n} \cdot p_n \right| &\leq F_l^{max} \quad \forall l \in CNEC \\
g_n^{min} &\leq g_n \leq g_n^{max} \quad \forall n
\end{align}

where $p_n$ is net position (injection minus withdrawal) at node $n$ and $F_l^{max}$ is the thermal limit of line $l$.

The critical advantage is that FBMC constraints reflect actual physical flows, allowing the market to exploit parallel paths and loop flows that ATC cannot represent.

\section{The Nordic Electricity Market}
\label{sec:nordic_market}

The Nordic electricity market, operated by Nord Pool, represents one of the world's most integrated multinational power markets. The market encompasses multiple bidding zones across Sweden (SE1--SE4), Norway (NO1--NO5), Finland (FI), and Denmark (DK1--DK2), each representing a region within which a single price applies during any given hour.

The generation mix across the Nordic region reflects diverse natural resource endowments. Norway and northern Sweden possess abundant hydropower resources, providing flexible and low-marginal-cost generation that dominates the supply stack during most hours. Sweden and Finland operate nuclear power plants that provide baseload generation with high availability factors. Wind power capacity has grown substantially across all Nordic countries over the past decade, introducing variability that increases the value of flexible transmission capacity.

Cross-border interconnection plays a central role in Nordic market operation. The region benefits from significant transmission capacity linking the Nordic synchronous area to continental Europe, the Baltic states, and between Nordic countries themselves. These interconnections enable arbitrage between regions with different generation mixes and demand patterns, contributing to price convergence and efficient resource utilization.

Demand patterns in the Nordic region exhibit pronounced seasonality. Winter peaks driven by electric heating substantially exceed summer demand, creating seasonal variation in both prices and transmission congestion. This seasonality proves critical for understanding FBMC benefits, as congestion---and thus the value of improved capacity allocation---concentrates in winter months.

Sweden's internal transmission system follows a predominantly north-south orientation, reflecting the geographic separation between hydroelectric generation in the north and population centers in the south. The SE2--SE3 corridor, with a nominal NTC of approximately 7,300~MW, represents the critical bottleneck constraining power transfer from the generation-rich northern zones to load-intensive SE3 and SE4. This corridor's frequent congestion makes it a natural focus for analyzing capacity allocation methodologies.

\section{Related Work}
\label{sec:related_work}

The academic literature on FBMC has grown substantially since the CWE implementation. Weinhold and Mieth~\cite{weinhold_pomato_2021} developed POMATO, an open-source tool for flow-based market analysis that has become a reference implementation. Their work demonstrated the importance of \gls{GSK} assumptions in determining FBMC outcomes.

Schönheit et al.~\cite{schonheit_gsk_2020} systematically analyzed how different GSK strategies affect the FBMC domain in CWE, finding that the choice of GSK methodology can significantly impact cross-border capacity availability.

Van den Bergh et al.~\cite{van_den_bergh_dc_2014} examined the validity of DC power flow assumptions in market models, providing theoretical justification for the approximations used in operational FBMC implementations.

The Central Western European (CWE) implementation, operational since May 2015, provides the most extensive empirical evidence on FBMC performance. Post-implementation analyses by ACER and the CWE TSOs documented welfare gains of 150--300 million EUR annually~\cite{cwe_tsos_2020}, though these estimates vary depending on the counterfactual ATC assumptions employed. Notably, the CWE experience revealed implementation challenges including the need for frequent PTDF matrix updates, coordination across multiple TSOs, and market participant adaptation to more complex price formation dynamics.

Beyond CWE, other European regions have pursued or are pursuing FBMC adoption. The Italy North capacity calculation region implemented flow-based allocation in 2021, while the Core capacity calculation region---encompassing Central Europe---completed its transition in June 2022~\cite{acer2022core}. These implementations provide comparative evidence, though direct comparison is complicated by differences in network topology, market structure, and congestion patterns.

Academic debate continues regarding optimal GSK methodologies. Schönheit et al.~\cite{schonheit2020gsk} demonstrated that different GSK strategies can yield PTDF variations of 10--20\% on critical network elements, materially affecting the flow-based domain. The choice between proportional, merit-order, and capacity-weighted GSKs remains an active research question, with operational implementations typically employing hybrid approaches refined through parallel running experience.

Nordic-specific research during the External Parallel Run (EPR) period from 2023--2024 remains limited in the public academic literature, though the Nordic RCC published periodic summary reports documenting welfare estimates and implementation progress. This thesis contributes to filling this gap by providing independent quantitative analysis using publicly available data.

\section{Summary}
\label{sec:background_summary}

This chapter has established the theoretical and institutional foundation for comparing FBMC and ATC capacity allocation. The fundamental distinction between the methodologies lies in their treatment of network physics: ATC abstracts the transmission grid into bilateral capacity limits that must be set conservatively to accommodate unknown flow patterns, while FBMC explicitly represents power flow distribution through PTDF matrices, enabling the market to identify feasible operating points that ATC would unnecessarily exclude.

The Nordic electricity market provides a compelling context for this comparison. Its combination of north-south internal congestion, substantial cross-border interconnection, and recent FBMC implementation creates conditions where the theoretical advantages of flow-based allocation can be empirically assessed. The existing academic literature, while demonstrating FBMC benefits in the Central Western European context, offers limited evidence specific to Nordic conditions---a gap this thesis addresses.


%%%%%%%%%%%%%%%%%%%%%%%%%%%%%%%%%%%%%%%%%%%%%%%%%%%%%%%%%%%%%%%%%%
%% CHAPTER 3: METHODS
%%%%%%%%%%%%%%%%%%%%%%%%%%%%%%%%%%%%%%%%%%%%%%%%%%%%%%%%%%%%%%%%%%
\cleardoublepage
\chapter{Methods}
\label{ch:methods}

This chapter describes the research methodology employed in this thesis. Section~\ref{sec:research_process} outlines the overall research process. Section~\ref{sec:model_formulation} presents the mathematical model formulation. Section~\ref{sec:data_collection} describes data sources. Section~\ref{sec:experimental_design} details the experimental design. Section~\ref{sec:validity_reliability} discusses validity and reliability considerations.

\section{Research Process}
\label{sec:research_process}

The research follows a quantitative simulation methodology with the following phases:

\begin{enumerate}
    \item \textbf{Model Development:} Formulation of zonal market clearing problem for both FBMC and ATC
    \item \textbf{Data Collection:} Acquisition of network, generation, demand, and price data from official sources
    \item \textbf{Implementation:} Python-based implementation using scipy optimization
    \item \textbf{Validation:} Comparison of model outputs against historical market outcomes
    \item \textbf{Analysis:} Systematic comparison of FBMC and ATC across scenarios
    \item \textbf{Interpretation:} Synthesis of findings and derivation of implications
\end{enumerate}

\section{Mathematical Model Formulation}
\label{sec:model_formulation}

\subsection{Network Representation}

The model employs a 13-node zonal representation comprising:
\begin{itemize}
    \item Four Swedish bidding zones: SE1, SE2, SE3, SE4
    \item Nine neighboring zones: NO1, NO3, NO4, FI, DK1, DK2, DE, PL, LT
\end{itemize}

The network includes 13 transmission corridors: three internal Swedish (SE1--SE2, SE2--SE3, SE3--SE4) and ten cross-border interconnectors.

\subsection{Objective Function}

Both FBMC and ATC formulations minimize total system cost:
\begin{equation}
\min \sum_{n \in N} \sum_{t \in T} C_n \cdot g_{n,t}
\end{equation}

where $N$ is the set of nodes, $T$ is the set of time periods, $C_n$ is marginal cost at node $n$, and $g_{n,t}$ is generation.

\subsection{Constraints}

\textbf{Power Balance:}
\begin{equation}
\sum_{u \in U_n} g_{u,t} - D_{n,t} = p_{n,t} \quad \forall n, t
\end{equation}

\textbf{Generation Limits:}
\begin{equation}
0 \leq g_{u,t} \leq G_u^{max} \quad \forall u, t
\end{equation}

\textbf{FBMC Flow Constraints:}
\begin{equation}
-F_l^{max} \leq \sum_{n} PTDF_{l,n} \cdot p_{n,t} \leq F_l^{max} \quad \forall l, t
\end{equation}

\textbf{ATC Flow Constraints:}
\begin{equation}
-NTC_l \leq f_{l,t} \leq NTC_l \quad \forall l, t
\end{equation}

\subsection{PTDF Calculation}

PTDFs are computed using the DC power flow approximation:
\begin{equation}
PTDF_{l,n} = \frac{1}{x_l} \left( \psi_{from(l),n} - \psi_{to(l),n} \right)
\end{equation}

where $x_l$ is line reactance and $\psi$ is the bus injection shift factor matrix derived from the network admittance matrix with SE3 as slack bus.

\subsection{Generation Shift Key Assumptions}

The calculation of zonal PTDFs requires assumptions about how generation within each bidding zone responds to changes in net position---the so-called Generation Shift Keys (GSKs). Different GSK strategies can materially affect the resulting flow-based domain and, consequently, market outcomes~\cite{schonheit_gsk_2020}.

This thesis employs an implicit proportional GSK assumption. By computing PTDFs directly from the DC power flow equations at the zonal level, generation shifts are distributed according to network admittance structure. Mathematically, this is equivalent to assuming that all generators within a zone respond proportionally to their electrical connectivity to the rest of the system.

This approach aligns with the ``flat start'' methodology used in early FBMC implementations and provides a conservative baseline for analysis. Alternative GSK strategies---such as merit-order based or capacity-weighted approaches---could yield different PTDF matrices and thus different estimates of FBMC benefits. However, sensitivity to GSK methodology primarily affects the absolute magnitude of cross-border capacity, not the directional finding that FBMC outperforms ATC under congested conditions.

A systematic comparison of GSK strategies is beyond the scope of this thesis but represents a valuable direction for future work, particularly as Nordic TSOs refine their operational GSK methodologies following the October 2024 go-live.

\textbf{Note:} The proportional GSK assumption represents a methodological choice with quantifiable impacts on results. Schönheit et al.~\cite{schonheit_gsk_2020} demonstrated that alternative GSK strategies can yield PTDF variations of 10--20\%, suggesting that sensitivity analysis across GSK methodologies would strengthen the conclusions presented here.

\section{Data Collection}
\label{sec:data_collection}

All data derives from authoritative public sources to ensure reproducibility:

\textbf{ENTSO-E Transparency Platform~\cite{entsoe_transparency_2024}:}
\begin{itemize}
    \item Day-ahead prices (document A44): Hourly prices for all bidding zones
    \item Actual total load (document A65): Hourly demand by zone
    \item Generation per type (document A75): Hourly generation by technology
\end{itemize}

\textbf{Svenska kraftnät~\cite{svk_kraftbalansen_2024}:}
\begin{itemize}
    \item Installed generation capacity by zone and technology
    \item Internal NTC values: SE1--SE2 (3,300 MW), SE2--SE3 (7,300 MW), SE3--SE4 (5,300 MW)
\end{itemize}

\textbf{PyPSA-Eur~\cite{horsch_pypsa_2018}:}
\begin{itemize}
    \item Transmission line electrical parameters (reactances) for PTDF calculation
\end{itemize}

\subsection{Marginal Cost Assumptions}

Generation marginal costs are assigned by technology:
\begin{itemize}
    \item Hydro: 0.5 EUR/MWh (water value approximation)
    \item Nuclear: 10 EUR/MWh
    \item Wind/Solar: 0 EUR/MWh
    \item CHP: 35--40 EUR/MWh
    \item Gas: 50 EUR/MWh
    \item Coal: 80 EUR/MWh
\end{itemize}

\section{Experimental Design}
\label{sec:experimental_design}

\subsection{Base Case Analysis}

The base case simulates one representative week (168 hours) from December 2024 under both FBMC and ATC regimes. December represents winter conditions with high demand and significant north-south price spreads.

\subsection{Multi-Period Analysis}

To capture seasonal variation, additional analyses are conducted for:
\begin{itemize}
    \item January 2024: Winter peak demand
    \item July 2024: Summer minimum demand
\end{itemize}

\subsection{Sensitivity Analysis}

Five sensitivity scenarios test robustness:
\begin{itemize}
    \item \textbf{Scenario A:} Continental prices +30\%
    \item \textbf{Scenario B:} Norwegian prices -30\%
    \item \textbf{Scenario C:} Price convergence (uniform 60 EUR/MWh)
    \item \textbf{Scenario D:} SE2--SE3 capacity -30\%
    \item \textbf{Scenario E:} Cross-border capacity +20\%
\end{itemize}

\section{Validity and Reliability}
\label{sec:validity_reliability}

\subsection{Internal Validity}

Internal validity concerns whether observed differences between FBMC and ATC outcomes can be attributed to the capacity allocation methodology rather than confounding factors. The simulation design ensures internal validity through strict experimental control: both FBMC and ATC scenarios use identical input data for demand, generation capacity, and marginal costs, differing only in how transmission constraints are formulated. This controlled comparison isolates the causal effect of the allocation methodology.

Additional measures supporting internal validity include systematic parameter variation in sensitivity analysis, which confirms that results respond predictably to input changes, and verification of optimization convergence for all scenarios, ensuring that reported outcomes represent true optima rather than numerical artifacts.

\subsection{External Validity}

External validity concerns whether findings generalize beyond the specific conditions analyzed. Several factors limit generalizability:

\begin{itemize}
    \item Zonal aggregation may not capture network effects occurring at finer spatial resolution
    \item Static NTC values throughout each simulation period do not reflect the hourly variation in actual TSO capacity calculations
    \item Marginal cost assumptions based on technology averages may differ from actual market bids, which reflect strategic considerations and opportunity costs
\end{itemize}

These limitations suggest that absolute magnitudes should be interpreted cautiously, while directional findings---FBMC outperforms ATC under congestion, benefits vary seasonally---are likely robust to modeling choices.

\subsection{Reliability}

Reliability concerns whether results are reproducible. The analysis supports reliability through several design choices: the HiGHS solver is a well-documented, open-source optimization tool with deterministic behavior; all input data derives from public sources (ENTSO-E Transparency Platform, Svenska kraftnät) enabling independent replication; and complete documentation of assumptions and parameters is provided in Appendix~\ref{app:data}.

\subsection{Validity of Zonal Aggregation}

A central methodological choice in this thesis is the use of a 13-node zonal model rather than a detailed nodal representation. This warrants explicit justification, as operational FBMC implementations rely on network models with thousands of nodes.

The distinction lies in the analytical objective. Transmission System Operators require high-resolution nodal models to ensure operational security: managing voltage profiles, satisfying N-1 contingency constraints, and coordinating real-time redispatch. This thesis addresses a different question---the economic efficiency of capacity allocation methodologies at the market coupling level. For this purpose, zonal aggregation is standard practice in the academic literature~\cite{weinhold_pomato_2021, schonheit_gsk_2020}.

The validity of zonal aggregation rests on whether the model captures the relevant physical mechanisms. The critical loop flow path (SE2--NO4--NO3--NO1--SE3) is explicitly represented in the 13-node topology. Furthermore, the close correspondence between simulated savings (€4.37M/week) and Nordic RCC official estimates (€6.1M/week average) provides empirical validation that the model captures economically significant dynamics despite its simplified representation. Section~\ref{sec:rcc_validation} provides quantitative validation against Nordic RCC operational data.


%%%%%%%%%%%%%%%%%%%%%%%%%%%%%%%%%%%%%%%%%%%%%%%%%%%%%%%%%%%%%%%%%%
%% CHAPTER 4: IMPLEMENTATION
%%%%%%%%%%%%%%%%%%%%%%%%%%%%%%%%%%%%%%%%%%%%%%%%%%%%%%%%%%%%%%%%%%
\cleardoublepage
\chapter{Implementation}
\label{ch:implementation}

This chapter describes the technical implementation of the market clearing model. Section~\ref{sec:software_tools} describes the software environment. Section~\ref{sec:ptdf_implementation} details PTDF matrix calculation. Section~\ref{sec:optimization_setup} presents the optimization implementation.

\section{Software Tools}
\label{sec:software_tools}

The model is implemented in Python 3.11 using the following libraries:
\begin{itemize}
    \item \texttt{scipy.optimize.linprog}: Linear programming solver interface
    \item \texttt{numpy}: Numerical array operations
    \item \texttt{pandas}: Data manipulation and analysis
    \item \texttt{HiGHS}: High-performance LP solver (via scipy)
\end{itemize}

The implementation follows the methodological framework documented in the POMATO tool~\cite{weinhold_pomato_2021}, though using a custom Python implementation rather than POMATO's Julia backend.

\section{PTDF Matrix Calculation}
\label{sec:ptdf_implementation}

The PTDF calculation proceeds in the following steps:

\begin{enumerate}
    \item Construct nodal admittance matrix $Y$ from line reactances
    \item Extract susceptance matrix $B$ (imaginary part of $Y$)
    \item Remove slack bus row/column to form reduced matrix $B'$
    \item Compute bus injection shift factors: $\Psi = (B')^{-1}$
    \item For each line $l$: $PTDF_{l,n} = \frac{1}{x_l}(\Psi_{from(l),n} - \Psi_{to(l),n})$
\end{enumerate}

SE3 is designated as the slack bus, representing the primary load center and price reference point.

Figure~\ref{fig:network_topology} shows the 13-node network topology used in the model.

\begin{figure}[!ht]
  \begin{center}
    \includegraphics[width=0.8\textwidth]{figures/network_topology_extended.png}
  \end{center}
  \caption{Network topology of the 13-node Nordic market model showing Swedish bidding zones (SE1--SE4) and neighboring market areas}
  \label{fig:network_topology}
\end{figure}

Figure~\ref{fig:ptdf_heatmap} presents the calculated PTDF matrix as a heatmap, showing how injections at each node affect flows on each transmission corridor.

\begin{figure}[!ht]
  \begin{center}
    \includegraphics[width=0.9\textwidth]{figures/ptdf_heatmap_v3.png}
  \end{center}
  \caption{Power Transfer Distribution Factor matrix showing percent of injected power at each node (columns) that flows through each transmission corridor (rows)}
  \label{fig:ptdf_heatmap}
\end{figure}

\section{Optimization Setup}
\label{sec:optimization_setup}

The market clearing problem is formulated in standard LP form:
\begin{equation}
\min_x c^T x \quad \text{s.t.} \quad A_{eq} x = b_{eq}, \quad A_{ub} x \leq b_{ub}, \quad lb \leq x \leq ub
\end{equation}

Decision variables $x$ include generation quantities for each unit and time period. The constraint matrices encode power balance (equality) and flow limits (inequality). The HiGHS solver is invoked through scipy's interface with default tolerances.

Typical solution times are under 10 seconds for a 168-hour simulation on standard hardware.


%%%%%%%%%%%%%%%%%%%%%%%%%%%%%%%%%%%%%%%%%%%%%%%%%%%%%%%%%%%%%%%%%%
%% CHAPTER 5: RESULTS AND ANALYSIS
%%%%%%%%%%%%%%%%%%%%%%%%%%%%%%%%%%%%%%%%%%%%%%%%%%%%%%%%%%%%%%%%%%
\cleardoublepage
\chapter{Results and Analysis}
\label{ch:results}

This chapter presents the results of the FBMC versus ATC comparison. Section~\ref{sec:base_case_results} presents base case findings. Section~\ref{sec:sensitivity_results} reports sensitivity analysis. Section~\ref{sec:seasonal_results} examines seasonal variation. Section~\ref{sec:mechanism_analysis} analyzes the underlying mechanisms.

\section{Base Case Results}
\label{sec:base_case_results}

Table~\ref{tab:base_case} summarizes the base case results for December 2024.

\begin{table}[ht]
\centering
\caption{Base Case Results: December 2024}
\label{tab:base_case}
\begin{tabular}{lrr}
\hline
\textbf{Metric} & \textbf{FBMC} & \textbf{ATC} \\
\hline
Total System Cost (M EUR) & -71.1 & -66.7 \\
Weekly Savings (M EUR) & \multicolumn{2}{c}{4.37} \\
Percentage Savings & \multicolumn{2}{c}{6.5\%} \\
SE2--SE3 Utilization & 88\% & 99\% \\
Average SE3 Price (EUR/MWh) & 42.3 & 48.7 \\
\hline
\end{tabular}
\end{table}

The negative costs reflect the model's treatment of demand as negative generation with associated welfare. FBMC achieves 6.5\% cost reduction compared to ATC, equivalent to 4.37 million EUR weekly savings.

The key observation is that FBMC reduces utilization on the SE2--SE3 corridor from 99\% to 88\% while achieving better overall outcomes. This occurs because FBMC can route some power through the NO4--NO3--NO1 loop path, utilizing cross-border capacity that ATC treats as independent.

Figure~\ref{fig:price_timeseries} shows the hourly price evolution under both methodologies for the base case period.

\begin{figure}[!ht]
  \begin{center}
    \includegraphics[width=0.95\textwidth]{figures/price_timeseries_v2.png}
  \end{center}
  \caption{Hourly price time series for Swedish bidding zones under FBMC and ATC}
  \label{fig:price_timeseries}
\end{figure}

Figure~\ref{fig:line_utilization} compares line utilization between FBMC and ATC across all transmission corridors.

\begin{figure}[!ht]
  \begin{center}
    \includegraphics[width=0.9\textwidth]{figures/line_utilization_comparison_v2.png}
  \end{center}
  \caption{Comparison of transmission line utilization between FBMC and ATC}
  \label{fig:line_utilization}
\end{figure}

\section{Sensitivity Analysis Results}
\label{sec:sensitivity_results}

Table~\ref{tab:sensitivity} presents sensitivity analysis results across five scenarios.

\begin{table}[htbp]
\centering
\caption{Sensitivity Analysis Results}
\label{tab:sensitivity}
\begin{tabular}{llrr}
\toprule
Scenario & Description & FBMC Savings (\%) & Savings (M EUR) \\
\midrule
Base & December 2024 conditions & 6.5 & 4.37 \\
A & Continental prices +30\% & 4.9 & 3.00 \\
B & Norwegian prices -30\% & 4.6 & 4.41 \\
C & Price convergence (60 EUR/MWh) & 0.0 & 0.00 \\
D & SE2--SE3 capacity -30\% & 16.1 & 8.02 \\
E & Cross-border capacity +20\% & 6.7 & 5.07 \\
\bottomrule
\end{tabular}
\end{table}

Figure~\ref{fig:sensitivity} visualizes these results. The sensitivity analysis reveals several important patterns that illuminate the mechanisms driving FBMC benefits.

Scenario C provides the most theoretically significant result. By imposing uniform prices across all zones, this scenario eliminates the arbitrage opportunities that motivate cross-border trade. The complete disappearance of FBMC benefits under price convergence confirms that the mechanism operates through enhanced trade facilitation: FBMC creates value by enabling welfare-improving transactions that ATC's conservative constraints would block. When no beneficial trades exist, improved constraint formulation provides no advantage.

Scenario D demonstrates that FBMC benefits amplify under system stress. Reducing SE2--SE3 capacity by 30\% increases the binding frequency of transmission constraints, creating more hours during which the choice of allocation methodology affects market outcomes. The resulting increase in FBMC savings from 6.5\% to 16.1\% indicates that flow-based allocation delivers greatest value precisely when efficient capacity utilization matters most---during periods of severe congestion.

Scenarios A and B test robustness to price variations in neighboring markets. Despite substantial changes to continental and Norwegian price levels, FBMC maintains positive benefits in both cases. This robustness suggests that the findings are not artifacts of the specific price conditions in the base case but reflect structural advantages of flow-based allocation that persist across plausible market conditions.

\begin{figure}[!ht]
  \begin{center}
    \includegraphics[width=0.9\textwidth]{figures/sensitivity_comparison.png}
  \end{center}
  \caption{FBMC cost savings across sensitivity scenarios}
  \label{fig:sensitivity}
\end{figure}

\section{Seasonal Variation Results}
\label{sec:seasonal_results}

Table~\ref{tab:seasonal} compares results across seasons.

\begin{table}[ht]
\centering
\caption{Seasonal Variation in FBMC Benefits}
\label{tab:seasonal}
\begin{tabular}{lrrr}
\hline
\textbf{Period} & \textbf{FBMC Savings} & \textbf{SE2--SE3 Util.} & \textbf{Avg. Spread} \\
\hline
January 2024 & 12.6\% & 92\% & 18.4 EUR/MWh \\
December 2024 & 6.5\% & 88\% & 12.1 EUR/MWh \\
July 2024 & 0.25\% & 53\% & 2.3 EUR/MWh \\
\hline
\end{tabular}
\end{table}

The results reveal striking seasonal variation in FBMC benefits, with winter savings exceeding summer savings by a factor of fifty. This pattern reflects the fundamental relationship between congestion and the value of improved capacity allocation.

During winter months, high heating demand drives substantial power consumption in southern Sweden, while hydroelectric and wind resources in the north create strong incentives for north-south transfer. The resulting pressure on the SE2--SE3 corridor pushes utilization rates above 90\%, creating frequent binding constraints where the choice of allocation methodology directly affects market outcomes. Under these conditions, FBMC's ability to exploit alternative routing through Norwegian interconnections yields material efficiency gains.

Summer conditions present a different picture. Reduced demand, combined with higher solar availability and lower heating loads, diminishes both absolute transfer needs and price differentials between zones. With the SE2--SE3 corridor operating at only 53\% utilization, transmission constraints rarely bind. When constraints do not bind, FBMC and ATC produce identical market outcomes---the sophisticated flow-based constraints simply do not activate.

Figure~\ref{fig:multi_period} provides a comprehensive visualization of these dynamics, illustrating the correlation between demand levels, congestion intensity, price spreads, and resulting FBMC benefits across the three analysis periods.

\begin{figure}[!ht]
  \begin{center}
    \includegraphics[width=0.95\textwidth]{figures/multi_period_comparison.png}
  \end{center}
  \caption{Multi-period comparison of FBMC versus ATC performance}
  \label{fig:multi_period}
\end{figure}

\section{Mechanism Analysis}
\label{sec:mechanism_analysis}

Analysis of the underlying mechanisms confirms the theoretical expectations:

\textbf{Topology Effect:} FBMC provides minimal benefit for Sweden's internal radial corridors (SE1--SE2--SE3--SE4) where power flow paths are uniquely determined. Benefits emerge only when cross-border connections create loops, specifically the SE2--NO4--NO3--NO1--SE3 path that allows alternative routing.

\textbf{Congestion Correlation:} Figure~\ref{fig:congestion_benefit} illustrates the positive relationship 
between SE2–SE3 corridor utilization and FBMC benefit magnitude across the 
four analysis scenarios. While the limited number of observations (n=4) 
precludes formal statistical inference, the monotonic pattern is consistent 
with theoretical expectations: FBMC benefits increase systematically with 
transmission utilization. The near-zero benefits during July (53% utilization) 
and substantial benefits under Scenario D (reduced capacity, higher 
utilization) bracket the relationship, demonstrating that FBMC value emerges 
only when transmission constraints actively bind.

\begin{figure}[!ht]
  \begin{center}
    \includegraphics[width=0.8\textwidth]{figures/congestion_benefit.png}
  \end{center}
  \caption{Relationship between SE2–SE3 corridor utilization and FBMC cost 
savings. Each point represents one analysis scenario (n=4): January 2024, 
December 2024, July 2024, and Scenario D (reduced capacity). The monotonic 
relationship supports the hypothesis that FBMC benefits increase with 
congestion levels, though the limited sample size precludes formal 
statistical inference.}
  \label{fig:congestion_benefit}
\end{figure}

\textbf{Arbitrage Mechanism:} The complete elimination of FBMC benefits under price convergence (Scenario C) confirms that the mechanism operates through enhanced arbitrage facilitation. FBMC creates value by enabling trades that ATC's conservative constraints would block.

Figure~\ref{fig:cross_border} shows the cross-border flow patterns that enable FBMC optimization.

\begin{figure}[!ht]
  \begin{center}
    \includegraphics[width=0.85\textwidth]{figures/cross_border_flows.png}
  \end{center}
  \caption{Cross-border flow patterns under FBMC and ATC}
  \label{fig:cross_border}
\end{figure}


%%%%%%%%%%%%%%%%%%%%%%%%%%%%%%%%%%%%%%%%%%%%%%%%%%%%%%%%%%%%%%%%%%
%% CHAPTER 6: DISCUSSION
%%%%%%%%%%%%%%%%%%%%%%%%%%%%%%%%%%%%%%%%%%%%%%%%%%%%%%%%%%%%%%%%%%
\cleardoublepage
\chapter{Discussion}
\label{ch:discussion}

This chapter interprets the results and discusses their broader implications.

\section{Interpretation of Results}

The findings of this thesis provide consistent support for all four hypotheses formulated in Chapter~\ref{ch:introduction}, collectively building a coherent picture of how and when FBMC delivers value relative to ATC.

The first hypothesis posited that FBMC would achieve positive welfare effects compared to ATC. The base case results confirm this unambiguously: under winter conditions, FBMC reduces total system costs by 6.5\%, equivalent to €4.37 million in weekly savings. This efficiency gain arises not from increasing total transmission capacity, but from allocating existing capacity more intelligently across the meshed network.

The second hypothesis concerned topology dependency---specifically, whether FBMC benefits require network loops to materialize. The analysis confirms this relationship. Sweden's internal corridors (SE1--SE2--SE3--SE4) follow a radial north-south configuration where power flow paths are uniquely determined by Kirchhoff's laws. For these corridors alone, FBMC offers no advantage over ATC. Benefits emerge only when cross-border interconnections create alternative routing possibilities, particularly the SE2--NO4--NO3--NO1--SE3 loop that allows congestion on the critical SE2--SE3 corridor to be partially relieved through Norwegian transmission infrastructure.

The third hypothesis predicted that FBMC benefits would amplify under system stress. Sensitivity Scenario D tested this by reducing SE2--SE3 capacity by 30\%, simulating a more constrained network. The result---FBMC savings increasing from 6.5\% to 16.1\%---confirms that flow-based allocation becomes most valuable precisely when efficient capacity utilization matters most. This finding has practical implications: FBMC implementation is not merely a theoretical improvement but delivers increasing returns during the high-congestion periods when market efficiency is most challenged.

The fourth hypothesis addressed the arbitrage mechanism underlying FBMC benefits. If FBMC creates value by enabling welfare-improving trades that ATC would block, then eliminating price differentials should eliminate benefits entirely. Scenario C tested this directly by imposing uniform prices across all zones. As predicted, FBMC savings dropped to exactly zero. This result confirms that the mechanism operates through enhanced arbitrage facilitation rather than any other channel, and provides confidence that the model correctly captures the economic fundamentals of capacity allocation.

Taken together, these findings demonstrate that FBMC delivers measurable, robust, and explainable efficiency gains for the Nordic market---gains that are largest when the system needs them most.

\section{Comparison with Literature}

The 6.5\% winter savings observed in this study exceeds typical estimates from CWE analyses, which generally report welfare improvements in the 1--3\% range~\cite{acer_fbmc_2019}. Several factors may explain this discrepancy.

The Nordic market experiences more severe baseline congestion than CWE on its critical corridors. The SE2--SE3 bottleneck operates near capacity limits during winter peaks with limited alternative routing options within Sweden, creating conditions where improved allocation methodology yields larger relative gains. The CWE region, with its more meshed internal topology and multiple parallel transmission paths, may already achieve relatively efficient capacity utilization under ATC.

Network topology differences also play a role. The Nordic configuration---with concentrated north-south flows and strategically positioned Norwegian interconnections---may create particularly favorable conditions for loop flow exploitation. The specific geometry of the SE2--NO4--NO3--NO1--SE3 path provides a clean alternative route that FBMC can leverage, whereas CWE's more complex topology may offer less distinct optimization opportunities.

Finally, modeling assumptions may contribute to the difference. This thesis uses static NTC values that may be more conservative than actual TSO practice, potentially overstating ATC's limitations. Additionally, the simplified 13-node representation excludes internal constraints that would reduce FBMC's optimization flexibility in a detailed model. These factors suggest that the 6.5\% figure should be interpreted as indicative of FBMC's potential under idealized conditions rather than a precise prediction of operational outcomes.

\section{Validation Against Nordic RCC Data}
\label{sec:rcc_validation}

The simulated results can be compared against official data from the Nordic Regional Coordination Centre (RCC), which coordinated an 18-month External Parallel Run (EPR) from early 2023 until the FBMC go-live in October 2024. During this period, both ATC and FBMC were computed in parallel, enabling direct comparison of market outcomes~\cite{nordic_rcc_epr_2024}.

The Nordic RCC reported average weekly welfare gains of approximately €6.1 million during the EPR period, with significant variation depending on congestion levels and seasonal conditions. Our simulation yields weekly savings of €4.37 million for December 2024 conditions, representing 72\% of the official estimate.

This discrepancy is consistent with known model simplifications. The simulation uses static NTC values, whereas operational ATC varies hourly based on TSO security assessments. The model includes only 13 CNECs compared to the hundreds monitored in operational FBMC. Additionally, the EPR average spans multiple seasons and congestion regimes, while the December simulation captures a single high-congestion period.

Importantly, the model captures the correct order of magnitude of FBMC benefits, suggesting that the 13-node zonal representation adequately reflects the economic mechanisms driving welfare improvements. The seasonal pattern identified in this thesis---with winter benefits substantially exceeding summer benefits---is also consistent with Nordic RCC observations that FBMC value concentrates in high-congestion periods.

\section{Policy Implications}

Several policy implications emerge:

\begin{enumerate}
    \item \textbf{Transmission Planning:} FBMC benefits are concentrated during high-congestion periods, suggesting targeted transmission investments may be cost-effective

    \item \textbf{Market Monitoring:} Regulators should track FBMC performance metrics seasonally, not just annually

    \item \textbf{Bidding Zone Review:} The strong north-south congestion pattern supports ongoing discussions about Swedish bidding zone configuration
\end{enumerate}

\section{Limitations}

Several limitations of this analysis warrant acknowledgment, as they affect interpretation of results and define the boundaries of valid inference.

The zonal aggregation employed in this thesis, while appropriate for analyzing market coupling economics, necessarily omits network effects occurring at finer spatial resolution. Intra-zonal congestion, voltage constraints, and detailed generator dispatch patterns are not captured. This limitation is partially mitigated by the close correspondence between simulated and official Nordic RCC results, but readers should recognize that operational FBMC involves complexities beyond this model's scope.

The temporal coverage of the analysis is limited to representative weeks from three periods in 2024. While these periods were selected to capture seasonal variation, they cannot represent the full distribution of market conditions occurring throughout a year. Unusual events---extreme weather, major outages, exceptional demand patterns---are not systematically analyzed.

Parameter assumptions introduce additional uncertainty. Static NTC values throughout each simulation period do not reflect hourly TSO adjustments based on real-time conditions. Marginal cost assumptions, while based on typical technology values, may diverge from actual market bids that incorporate strategic considerations, start-up costs, and opportunity costs not captured in simple merit-order assumptions.

The NTC values employed as the ATC baseline represent maximum announced capacities, whereas operational ATC frequently involves reliability-driven reductions. This comparison methodology may overstate FBMC benefits relative to a baseline using actual historical ATC allocations. However, the use of maximum NTC also represents a ``best case'' for ATC, as any reduction from these maxima would further constrain ATC outcomes while leaving FBMC's flow-based domain largely unaffected.

Finally, the model assumes competitive bidding behavior with all participants bidding at marginal cost. Strategic behavior---including potential exercise of market power or portfolio optimization across multiple zones---is not modeled. To the extent that actual market outcomes reflect strategic considerations, simulated results may diverge from realized performance.

\section{Reflections on Sustainability and Ethics}
\label{sec:sustainability_ethics}

In accordance with the requirements of the Swedish Higher Education Ordinance, this section reflects on the sustainability implications and ethical dimensions of the research conducted in this thesis.

\subsection{Sustainability Analysis}

The transition to sustainable energy systems represents one of the defining challenges of the 21st century. Electricity market design, while often viewed as a purely economic or technical concern, has profound implications for the pace and equity of this transition. This thesis contributes to sustainability objectives through its analysis of capacity allocation methodologies that enable more efficient utilization of existing transmission infrastructure.

\subsubsection{Ecological Sustainability and Climate Action (SDG 7, SDG 13)}

Flow-Based Market Coupling directly supports the United Nations Sustainable Development Goal 7 (Affordable and Clean Energy) and Goal 13 (Climate Action) through two primary mechanisms.

First, FBMC enables higher utilization of cross-border transmission capacity, reducing the need for new transmission line construction. Transmission infrastructure development carries significant environmental costs: land use changes, habitat fragmentation, and the embodied carbon in steel towers and aluminum conductors. By extracting greater value from existing infrastructure, FBMC represents a form of resource efficiency that delays or avoids these environmental impacts.

Second, and more significantly, improved market coupling facilitates the integration of variable renewable energy sources. The Nordic region's abundant wind and hydroelectric resources are geographically concentrated in areas distant from major load centers. Wind power in northern Sweden and Norway, hydroelectric reservoirs in the Scandinavian mountains---these resources require robust transmission access to reach consumers in southern population centers. When transmission constraints bind under ATC, renewable generation may be curtailed or displaced by fossil-fueled alternatives closer to load. FBMC's ability to identify non-obvious transmission paths reduces such curtailment, directly contributing to decarbonization objectives.

The seasonal pattern identified in this thesis reinforces this connection. Winter periods---when FBMC delivers its greatest benefits---coincide with peak heating demand and maximum pressure on the power system. These are precisely the conditions under which inefficient capacity allocation would most likely force reliance on carbon-intensive peaking generation. The 6.5\% cost reduction achieved by FBMC during winter conditions thus represents not merely economic savings but avoided emissions.\footnote{The computational requirements of the research itself---approximately 50 CPU-hours for all simulations using the HiGHS solver---represent a modest environmental footprint compared to the potential emissions reductions enabled by improved market design.}

\subsubsection{Economic Sustainability (SDG 8, SDG 9)}

Sustainable Development Goal 8 (Decent Work and Economic Growth) and Goal 9 (Industry, Innovation, and Infrastructure) are served by market designs that promote economic efficiency and optimal infrastructure utilization.

The welfare gains quantified in this thesis---€4.37 million weekly under winter conditions---represent resources that can be redirected toward productive economic activity rather than being dissipated through inefficient dispatch. Over annual timescales, these savings accumulate to substantial sums that benefit electricity consumers, reduce industrial energy costs, and improve the competitiveness of Nordic economies.

Furthermore, FBMC exemplifies the principle of smart infrastructure management. Rather than responding to transmission constraints solely through capital-intensive grid expansion, flow-based allocation represents an operational innovation that extracts latent value from existing assets. This approach aligns with circular economy principles and demonstrates that software and algorithmic improvements can substitute for physical resource consumption.

\subsubsection{Social Sustainability (SDG 10, SDG 11)}

The social dimensions of electricity market design are subtle but consequential. Sustainable Development Goal 10 (Reduced Inequalities) and Goal 11 (Sustainable Cities and Communities) require consideration of how market efficiency gains are distributed across society.

FBMC's efficiency improvements manifest as reduced wholesale electricity prices, benefits that flow through to end consumers. However, the distribution of these benefits is not uniform. Industrial consumers with direct market exposure capture savings immediately, while residential consumers may experience attenuated benefits depending on retail market structures and regulatory pass-through requirements. Future research should examine the distributional equity of FBMC benefits across consumer categories and geographic regions.

A potential concern is that optimized market algorithms could systematically disadvantage certain regions. If FBMC consistently routes power away from particular corridors to exploit loop flows, communities along those corridors might experience reduced grid investment or reliability concerns. The analysis in this thesis does not identify such patterns, but ongoing monitoring of regional outcomes is warranted as FBMC operation matures.

\subsection{Ethical Considerations}

\subsubsection{Research Ethics}

This thesis adheres to established principles of research integrity and scientific ethics. Several specific measures warrant discussion.

Data transparency and reproducibility constitute foundational ethical obligations in computational research. All input data used in this thesis derives from publicly accessible sources: the ENTSO-E Transparency Platform for demand and generation data, Svenska kraftnät publications for network parameters, and Nord Pool for price information. This reliance on open data ensures that the analysis can be independently verified and extended by other researchers, fulfilling the scientific community's expectation of reproducibility.

The simulation code developed for this thesis employs open-source tools (Python, HiGHS solver) rather than proprietary software, further supporting reproducibility. While the specific implementation code is not published as part of this thesis, the methodological descriptions in Chapters 3 and 4 provide sufficient detail for independent replication.

No human subjects were involved in this research, and no personal data was processed. The market data analyzed represents aggregated, anonymized information about system-level outcomes rather than individual consumer behavior. Consequently, no ethical review board approval was required, and no GDPR compliance concerns arise.

\subsubsection{Technology Ethics}

Beyond research ethics, the responsible engineer must consider the broader societal implications of the technologies they develop and analyze. Several technology ethics dimensions are relevant to electricity market coupling.

Algorithmic transparency presents an emerging concern in market design. As market clearing algorithms become more sophisticated---incorporating flow-based constraints, machine learning forecasts, and complex optimization routines---the ability of market participants and regulators to understand and scrutinize market outcomes may diminish. FBMC, while more physically accurate than ATC, is also more opaque: the relationship between bids, network constraints, and clearing prices is mediated by PTDF matrices and RAM calculations that few market participants fully comprehend. This opacity could erode trust in market institutions or create opportunities for sophisticated actors to exploit informational advantages.

Market power and competitive effects deserve consideration. By expanding the effective transmission capacity between zones, FBMC intensifies competition among generators across wider geographic areas. This generally benefits consumers through lower prices but may disadvantage generators in high-cost regions who previously enjoyed protection from import competition. The transition to FBMC thus creates winners and losers, raising questions of procedural fairness regarding how such transitions are governed and whether affected parties receive adequate notice and adjustment support.

Finally, the automation of critical infrastructure decisions carries inherent risks. Market coupling algorithms clear billions of euros in transactions daily, and errors or manipulation could have severe economic consequences. The Nordic TSOs and SDAC (Single Day-Ahead Coupling) operators maintain extensive safeguards against such risks, but the concentration of decision-making authority in algorithmic systems warrants ongoing vigilance.

\subsection{Summary}

This thesis contributes to sustainability objectives by analyzing market mechanisms that enable more efficient utilization of transmission infrastructure and facilitate renewable energy integration. The research adheres to ethical standards of data transparency and reproducibility. Broader technology ethics considerations---including algorithmic transparency, competitive effects, and automation risks---are acknowledged as important dimensions of electricity market design that merit continued attention from researchers, regulators, and market participants.


%%%%%%%%%%%%%%%%%%%%%%%%%%%%%%%%%%%%%%%%%%%%%%%%%%%%%%%%%%%%%%%%%%
%% CHAPTER 7: CONCLUSIONS AND FUTURE WORK
%%%%%%%%%%%%%%%%%%%%%%%%%%%%%%%%%%%%%%%%%%%%%%%%%%%%%%%%%%%%%%%%%%
\cleardoublepage
\chapter{Conclusions and Future Work}
\label{ch:conclusions}

\section{Conclusions}
\label{sec:conclusions}

This thesis set out to quantify the efficiency differential between Flow-Based Market Coupling and Available Transfer Capacity for the Nordic electricity market. Through simulation of market clearing under both regimes using empirical data from the Swedish transmission system, it has produced findings that address each of the research questions posed in Chapter~\ref{ch:introduction}.

The central finding is that FBMC provides measurable and economically significant efficiency gains over ATC, but these gains are neither uniform nor guaranteed. Under winter conditions with high north-south congestion, FBMC achieves cost savings of 6.5\% compared to ATC---a substantial improvement that validates the Nordic region's decision to implement flow-based allocation. However, during summer periods with low demand and minimal congestion, FBMC benefits shrink to just 0.25\%, with benefits varying by more than an order of magnitude between seasons. This pattern reveals that FBMC is not a universal improvement but rather a tool whose value is contingent on system conditions.

The thesis identifies the physical mechanism enabling these benefits: the presence of cross-border loop flows that create alternative routing paths for power delivery. Sweden's internal transmission corridors are predominantly radial, offering no scope for FBMC optimization. Benefits materialize only when Norwegian interconnections provide parallel paths that can relieve congestion on the critical SE2--SE3 bottleneck. This topology dependency has implications for transmission planning---investments that create network loops may yield compounding benefits by expanding the scope for flow-based optimization.

The robustness of these findings has been confirmed through sensitivity analysis across five scenarios spanning price variations and capacity constraints. FBMC maintains positive benefits across all scenarios except complete price convergence, where---as theory predicts---the value of improved allocation disappears entirely. Under constrained transmission conditions, benefits amplify to 16.1\%, demonstrating that FBMC delivers greatest value during periods of system stress when efficient allocation matters most.

These conclusions carry implications for Nordic market design and regulatory monitoring. The strong seasonal pattern suggests that annual averages obscure significant temporal variation in FBMC performance, and that regulators should track benefits on a seasonal or monthly basis. The topology dependency implies that Swedish bidding zone configuration and cross-border interconnection strategy should be evaluated jointly with capacity allocation methodology. Finally, the validation of simulated results against Nordic RCC operational data provides confidence that simplified zonal models can capture the economically significant dynamics of flow-based market coupling.

These Nordic findings contribute to the broader European evidence base on FBMC implementation. The Core region experience since June 2022 provides complementary evidence from a more densely meshed network, while ongoing discussions about extending FBMC to additional capacity calculation regions can draw on the Nordic demonstration that flow-based allocation delivers value even in predominantly radial topologies.

\section{Future Work}
\label{sec:future_work}

Several directions for future work are identified:

\begin{enumerate}
    \item \textbf{Extended temporal analysis:} Full annual simulation to capture all seasonal patterns
    \item \textbf{Validation against actual market data:} Comparison with Nordic RCC parallel run results
    \item \textbf{Higher resolution modeling:} Transition from zonal to nodal representation
    \item \textbf{Integration with N-1 security:} Explicit modeling of contingency constraints
    \item \textbf{Welfare distribution analysis:} Examination of how FBMC benefits are distributed across market participants
\end{enumerate}


%%%%%%%%%%%%%%%%%%%%%%%%%%%%%%%%%%%%%%%%%%%%%%%%%%%%%%%%%%%%%%%%%%
%% BIBLIOGRAPHY
%%%%%%%%%%%%%%%%%%%%%%%%%%%%%%%%%%%%%%%%%%%%%%%%%%%%%%%%%%%%%%%%%%
\cleardoublepage

\iftoggle{biblatex}{
    \printbibliography[heading=bibintoc, title=References]
}{
    \renewcommand{\bibname}{References}
    \addcontentsline{toc}{chapter}{References}
    \bibliography{references}
}


%%%%%%%%%%%%%%%%%%%%%%%%%%%%%%%%%%%%%%%%%%%%%%%%%%%%%%%%%%%%%%%%%%
%% APPENDICES
%%%%%%%%%%%%%%%%%%%%%%%%%%%%%%%%%%%%%%%%%%%%%%%%%%%%%%%%%%%%%%%%%%
\cleardoublepage
\appendix
\renewcommand{\chaptermark}[1]{\markboth{Appendix \thechapter\relax:\thinspace\relax#1}{}}

\chapter{PTDF Matrix for Swedish Network}
\label{app:ptdf}

Table~\ref{tab:ptdf_excerpt} shows an excerpt of the calculated PTDF matrix for key corridors.

\begin{table}[ht]
\centering
\caption{PTDF Matrix Excerpt (Selected Corridors)}
\label{tab:ptdf_excerpt}
\begin{tabular}{lrrrr}
\hline
\textbf{Corridor} & \textbf{SE1} & \textbf{SE2} & \textbf{SE3} & \textbf{SE4} \\
\hline
SE1--SE2 & 1.000 & 0.000 & 0.000 & 0.000 \\
SE2--SE3 & 0.714 & 0.714 & 0.000 & 0.000 \\
SE3--SE4 & 0.286 & 0.286 & 0.286 & 0.000 \\
SE2--NO4 & 0.286 & 0.286 & 0.000 & 0.000 \\
\hline
\end{tabular}
\end{table}

The non-trivial PTDF values for SE2--NO4 (0.286) indicate the presence of loop flow paths that enable FBMC optimization.

\chapter{Data Sources and Parameters}
\label{app:data}

Complete parameter values and data sources are documented in this appendix for reproducibility.

\section{Generation Capacity by Zone}

\begin{table}[ht]
\centering
\caption{Installed Generation Capacity (MW)}
\label{tab:capacity}
\begin{tabular}{lrrrr}
\hline
\textbf{Technology} & \textbf{SE1} & \textbf{SE2} & \textbf{SE3} & \textbf{SE4} \\
\hline
Hydro & 5,200 & 8,100 & 1,800 & 200 \\
Nuclear & 0 & 0 & 6,900 & 0 \\
Wind & 2,100 & 4,800 & 3,200 & 1,600 \\
CHP & 300 & 400 & 2,100 & 1,200 \\
\hline
\end{tabular}
\end{table}

Source: Svenska kraftnät Kraftbalansen 2024~\cite{svk_kraftbalansen_2024}

\label{pg:lastPageofMainmatter}

\clearpage
\section*{For DIVA}
\divainfo{pg:lastPageofPreface}{pg:lastPageofMainmatter}

\end{document}
