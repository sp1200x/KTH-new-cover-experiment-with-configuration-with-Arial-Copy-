%%
%% KTH Master's Thesis: FBMC vs ATC for Nordic Electricity Market
%% Migrated to new KTH template
%%

\RequirePackage{ifxetex}
\RequirePackage{ifluatex}
\newif\ifxeorlua
\ifxetex\xeorluatrue\fi
\ifluatex\xeorluatrue\fi

\ifxeorlua
\RequirePackage{expl3}
\RequirePackage{etoolbox}
\ExplSyntaxOn
\pdf_version_gset:n{1.5}
\ExplSyntaxOff
\else
\RequirePackage{expl3}
\ExplSyntaxOn
\pdf_version_gset:n{1.5}
\ExplSyntaxOff
\fi

\makeatletter
\newcommand{\disablepackage}[2]{%
  \disable@package@load{#1}{#2}%
}
\newcommand{\reenablepackage}[1]{%
  \reenable@package@load{#1}%
}
\makeatother
\ifxeorlua
\disablepackage{transparent}{}
\fi

\documentclass[
english,
bibtex,
]{kththesis}

\iftoggle{biblatex}{
    \usepackage[style=ieee,citestyle=numeric-comp]{biblatex}
    \addbibresource{references.bib}
}{
    \bibliographystyle{bibstyle/myIEEEtran}
}

\input{lib/includes}
\input{lib/kthcolors}
\input{lib/defines}

\makeatletter
\let\verbatimsc\@undefined
\let\endverbatimsc\@undefined
\lst@AddToHook{Init}{\hyphenpenalty=50\relax}
\makeatother

\lstnewenvironment{verbatimsc}
    {
    \lstset{%
        basicstyle=\ttfamily\tiny,
        backgroundcolor=\color{white},
        columns=[l]fixed,
        language=[LaTeX]TeX,
        keywordstyle=\color{red},
        breaklines=true,
        breakatwhitespace=true,
        breakindent=0em,
        frame=none,
        postbreak={}
    }
}{}

\newcolumntype{L}[1]{>{\raggedright\let\newline\\\arraybackslash\hspace{0pt}}p{#1}}

\iftoggle{biblatex}{
    \usepackage[plainpages=false]{hyperref}
}{
    \usepackage[
    backref=page,
    pagebackref=false,
    plainpages=false,
    unicode=true,
    bookmarks=true,
    bookmarksopen=false,
    pdfpagemode=UseNone,
    destlabel,
    pdfencoding=auto,
    ]{hyperref}
}

\usepackage[all]{hypcap}

\usepackage[acronym, style=super, section=section, nonumberlist, nomain,
nopostdot]{glossaries}
\setlength{\glsdescwidth}{0.75\textwidth}
\usepackage[]{glossaries-extra}

\newglossary[tlg]{readme}{tld}{tdn}{README acronyms}

\input{lib/includes-after-hyperref}

\makeglossaries

\ifxeorlua
%%% Local Variables:
%%% mode: latex
%%% TeX-master: t
%%% End:
% The following command is used with glossaries-extra
\setabbreviationstyle[acronym]{long-short}
% The form of the entries in this file is \newacronym{label}{acronym}{phrase}
%                                      or \newacronym[options]{label}{acronym}{phrase}
% see "User Manual for glossaries.sty" for the  details about the options, one example is shown below
% note the specification of the long form plural in the line below
\newacronym[longplural={Debugging Information Entities}]{DIE}{DIE}{Debugging Information Entity}
%
% The following example also uses options
\newacronym[shortplural={OSes}, firstplural={operating systems (OSes)}]{OS}{OS}{operating system}

% note the use of a non-breaking dash in long text for the following acronym
\newacronym{IQL}{IQL}{Independent Q‑Learning}

% example of putting in a trademark on first expansion
\newacronym[first={NVIDIA OpenSHMEM Library (NVSHMEM\texttrademark)}]{NVSHMEM}{NVSHMEM}{NVIDIA OpenSHMEM Library}

\newacronym{API}{API}{Application Programming Interface}
\newacronym{DiVA}{DiVA}{Digitala Vetenskapliga Arkivet}
\newacronym{IMRAD}{IMRAD}{Introduction, Methods, Results, and Discussione}
\newacronym{JSON}{JSON}{JavaScript Object Notation}
\newacronym{KI}{KI}{Karolinska Institutet}
\newacronym{KOPPS}{KOPPS}{Kurs- och programplaneringssystemet}
\newacronym{LADOK}{LADOK}{Lokalt adb–baserat dokumentationssystemt}
\newacronym{TIMTM}{TIMTM}{Interactive Media Technology}
\newacronym{TMMTM}{TMMTM}{Media Managemen}
\newacronym{KTH}{KTH}{KTH Royal Institute of Technology}

\newacronym{LAN}{LAN}{Local Area Network}
\newacronym{VM}{VM}{virtual machine}
% note the use of a non-breaking dash in the following acronym
\newacronym{WiFi}{Wi‑Fi}{Wireless Fidelity}

\newacronym{WLAN}{WLAN}{Wireless Local Area Network}
\newacronym{UN}{UN}{United Nations}
\newacronym{SDG}{SDG}{Sustainable Development Goal}

% FBMC vs ATC Thesis Acronyms
% Core methodology
\newacronym{FBMC}{FBMC}{Flow-Based Market Coupling}
\newacronym{ATC}{ATC}{Available Transfer Capacity}
\newacronym{NTC}{NTC}{Net Transfer Capacity}
\newacronym{PTDF}{PTDF}{Power Transfer Distribution Factor}
\newacronym{LODF}{LODF}{Line Outage Distribution Factor}
\newacronym{CNEC}{CNEC}{Critical Network Element with Contingency}
\newacronym{RAM}{RAM}{Remaining Available Margin}
\newacronym{GSK}{GSK}{Generation Shift Key}

% Organizations and regulations
\newacronym{CACM}{CACM}{Capacity Allocation and Congestion Management}
\newacronym[longplural={European Network of Transmission System Operators for Electricity}]{ENTSO-E}{ENTSO-E}{European Network of Transmission System Operators for Electricity}
\newacronym[shortplural={TSOs}]{TSO}{TSO}{Transmission System Operator}
\newacronym{RCC}{RCC}{Regional Coordination Centre}
\newacronym{ACER}{ACER}{Agency for the Cooperation of Energy Regulators}

% Regions
\newacronym{CWE}{CWE}{Central Western Europe}
\newacronym{EU}{EU}{European Union}

% Technical
\newacronym{DC}{DC}{Direct Current}
\newacronym{AC}{AC}{Alternating Current}
\newacronym{OPF}{OPF}{Optimal Power Flow}
\newacronym{LP}{LP}{Linear Programming}

\else
\input{lib/acronyms-for-pdflatex}
\fi

\input{lib/placeHolder_lbx_files}
\input{custom_configuration}
\IfFileExists{custom_configuration_plaintext.tex}{\input{custom_configuration_plaintext.tex}}{}

\EnglishKeywords{Flow-Based Market Coupling, Available Transfer Capacity, Nordic Electricity Market, Power Transfer Distribution Factors, Congestion Management, Zonal Pricing, Market Efficiency}
\SwedishKeywords{Flödesbaserad marknadskoppling, Tillgänglig överföringskapacitet, Nordisk elmarknad, Effektöverföringsfaktorer, Trängselhantering, Zonprissättning, Marknadseffektivitet}

\presentationDateAndTimeISO{2025-06-15 10:00}
\presentationLanguage{eng}
\presentationRoom{TBD}
\presentationAddress{Teknikringen 33}
\presentationCity{Stockholm}
\opponentsNames{TBD}

\trita{TRITA -- EECS-EX}{2025:0000}

\input{lib/pdf_related_includes}

\hypersetup{
	colorlinks  = true,
	breaklinks  = true,
	linkcolor   = \linkscolor,
	urlcolor    = \urlscolor,
	citecolor   = \refscolor,
	anchorcolor = black
}

\ifnomenclature
\renewcommand*{\pagedeclaration}[1]{\unskip, \dotfill\hyperlink{page.#1}{page\nobreakspace#1}}
\renewcommand{\nomname}{List of Symbols Used}
\renewcommand{\nompreamble}{The following symbols will be later used within the body of the thesis.}
\makenomenclature
\fi

\usepackage{subfiles}

\listfiles

\begin{document}
\selectlanguage{english}

\pagenumbering{alph}
\kthcover
\clearpage\thispagestyle{empty}\mbox{}
\titlepage
\bookinfopage

\frontmatter
\setcounter{page}{1}

%%%%%%%%%%%%%%%%%%%%%%%%%%%%%%%%%%%%%%%%%%%%%%%%%%%%%%%%%%%%%%%%%%
%% ENGLISH ABSTRACT
%%%%%%%%%%%%%%%%%%%%%%%%%%%%%%%%%%%%%%%%%%%%%%%%%%%%%%%%%%%%%%%%%%
\begin{abstract}
\markboth{\abstractname}{}
\begin{scontents}[store-env=lang]
eng
\end{scontents}

\begin{scontents}[store-env=abstracts,print-env=true]
The European electricity market is undergoing a fundamental transformation in cross-border transmission capacity allocation, with Flow-Based Market Coupling (FBMC) replacing the traditional Available Transfer Capacity (ATC) methodology. This thesis presents a quantitative comparison of FBMC and ATC for the Nordic electricity market, focusing on the Swedish transmission system following the October 2024 FBMC implementation.

A 13-node zonal market model was developed representing the four Swedish bidding zones (SE1--SE4) and nine neighboring market areas. The model implements DC power flow-based Power Transfer Distribution Factor (PTDF) calculations and linear programming optimization using Python with the HiGHS solver. Market clearing was simulated under both FBMC (PTDF-constrained) and ATC (NTC-limited) regimes using data from the ENTSO-E Transparency Platform and Svenska kraftnät.

The results demonstrate that FBMC produces measurable efficiency gains over ATC, but with significant seasonal variation. During winter conditions (December 2024), FBMC achieved 6.5\% cost savings (4.37 million EUR weekly) compared to ATC, driven by high congestion on the SE2--SE3 corridor (88\% utilization). In contrast, summer conditions (July 2024) showed only 0.25\% savings due to reduced congestion (53\% utilization), representing a 50-fold seasonal differential.

Sensitivity analysis across five scenarios confirmed the robustness of these findings: FBMC benefits increase under constrained transmission (16.1\% savings with 30\% SE2--SE3 capacity reduction) and disappear entirely under price convergence conditions, validating that the mechanism operates through enhanced arbitrage facilitation. The topology analysis revealed that FBMC benefits are negligible for Sweden's internal radial corridors but become significant when cross-border connections create network loops enabling flow redistribution.

These findings have implications for Nordic transmission planning and demonstrate that FBMC's value proposition is primarily realized during periods of high congestion, suggesting targeted implementation strategies may be appropriate.
\end{scontents}

\subsection*{Keywords}
\begin{scontents}[store-env=keywords,print-env=true]
\InsertKeywords{english}
\end{scontents}
\end{abstract}
\cleardoublepage

%%%%%%%%%%%%%%%%%%%%%%%%%%%%%%%%%%%%%%%%%%%%%%%%%%%%%%%%%%%%%%%%%%
%% SWEDISH ABSTRACT (Sammanfattning)
%%%%%%%%%%%%%%%%%%%%%%%%%%%%%%%%%%%%%%%%%%%%%%%%%%%%%%%%%%%%%%%%%%
\babelpolyLangStart{swedish}
\begin{abstract}
\markboth{\abstractname}{}
\begin{scontents}[store-env=lang]
swe
\end{scontents}

\begin{scontents}[store-env=abstracts,print-env=true]
Den europeiska elmarknaden genomgår en fundamental förändring i hur gränsöverskridande överföringskapacitet allokeras, där flödesbaserad marknadskoppling (FBMC) ersätter den traditionella metoden med tillgänglig överföringskapacitet (ATC). Detta examensarbete presenterar en kvantitativ jämförelse mellan FBMC och ATC för den nordiska elmarknaden, med fokus på det svenska transmissionssystemet efter FBMC-implementeringen i oktober 2024.

En 13-nodsmodell utvecklades för att representera de fyra svenska elområdena (SE1--SE4) samt nio angränsande marknadsområden. Modellen implementerar DC-lastflödesbaserade beräkningar av effektöverföringsfaktorer (PTDF) och linjärprogrammeringsoptimering med Python och HiGHS-lösaren. Marknadsklarering simulerades under både FBMC (PTDF-begränsad) och ATC (NTC-begränsad) med data från ENTSO-E Transparency Platform och Svenska kraftnät.

Resultaten visar att FBMC ger mätbara effektivitetsvinster jämfört med ATC, men med betydande säsongsvariation. Under vinterförhållanden (december 2024) uppnådde FBMC 6,5\% kostnadsbesparingar (4,37 miljoner EUR per vecka) jämfört med ATC, drivet av hög trängsel på SE2--SE3-korridoren (88\% utnyttjande). Under sommarförhållanden (juli 2024) var besparingarna endast 0,25\% på grund av minskad trängsel (53\% utnyttjande), vilket representerar en 50-faldig säsongsskillnad.

Känslighetsanalys över fem scenarier bekräftade robustheten i dessa resultat: FBMC-fördelarna ökar vid begränsad transmission (16,1\% besparing vid 30\% SE2--SE3-kapacitetsreduktion) och försvinner helt vid priskonvergens, vilket validerar att mekanismen fungerar genom förbättrad arbitragefacilitering.
\end{scontents}

\subsection*{Nyckelord}
\begin{scontents}[store-env=keywords,print-env=true]
\InsertKeywords{swedish}
\end{scontents}
\end{abstract}
\cleardoublepage
\babelpolyLangStop{swedish}

%%%%%%%%%%%%%%%%%%%%%%%%%%%%%%%%%%%%%%%%%%%%%%%%%%%%%%%%%%%%%%%%%%
%% ACKNOWLEDGMENTS
%%%%%%%%%%%%%%%%%%%%%%%%%%%%%%%%%%%%%%%%%%%%%%%%%%%%%%%%%%%%%%%%%%
\section*{Acknowledgments}
\markboth{Acknowledgments}{}

I would like to express my sincere gratitude to my supervisor, Associate Professor Mohammad Reza Hesamzadeh, for his guidance and expertise in electricity market design throughout this project.

I am grateful to Svenska kraftnät and ENTSO-E for making comprehensive market data publicly available through their transparency platforms, enabling independent research on European electricity markets.

Finally, I thank my family and friends for their support during my studies at KTH.

\acknowlegmentssignature

\fancypagestyle{plain}{}
\renewcommand{\chaptermark}[1]{ \markboth{#1}{}}
\tableofcontents
\markboth{\contentsname}{}

\cleardoublepage
\listoffigures

\cleardoublepage
\listoftables

\cleardoublepage
\newglossarystyle{mylong}{%
  \setglossarystyle{long}%
  \renewenvironment{theglossary}%
     {\begin{longtable}[l]{@{}p{\dimexpr 2cm-\tabcolsep}p{0.8\hsize}}}%
     {\end{longtable}}%
 }
\printglossary[style=mylong, type=\acronymtype, title={List of acronyms and abbreviations}]

\ifnomenclature
    \cleardoublepage
    \printnomenclature
\fi

\label{pg:lastPageofPreface}

%%%%%%%%%%%%%%%%%%%%%%%%%%%%%%%%%%%%%%%%%%%%%%%%%%%%%%%%%%%%%%%%%%
%% MAIN MATTER
%%%%%%%%%%%%%%%%%%%%%%%%%%%%%%%%%%%%%%%%%%%%%%%%%%%%%%%%%%%%%%%%%%
\mainmatter
\glsresetall
\renewcommand{\chaptermark}[1]{\markboth{#1}{}}
\selectlanguage{english}

%%%%%%%%%%%%%%%%%%%%%%%%%%%%%%%%%%%%%%%%%%%%%%%%%%%%%%%%%%%%%%%%%%
%% CHAPTER 1: INTRODUCTION
%%%%%%%%%%%%%%%%%%%%%%%%%%%%%%%%%%%%%%%%%%%%%%%%%%%%%%%%%%%%%%%%%%
\chapter{Introduction}
\label{ch:introduction}

The efficient allocation of cross-border transmission capacity is fundamental to the operation of interconnected electricity markets. As European power systems become increasingly integrated, the methodology used for capacity allocation directly impacts market efficiency, price formation, and ultimately consumer welfare. This chapter introduces the context, problem statement, and objectives of this thesis.

\section{Background}
\label{sec:background}

The European Union's internal electricity market represents one of the world's largest integrated power systems, spanning multiple synchronous areas and dozens of bidding zones. The efficient operation of this market depends critically on how transmission capacity between bidding zones is allocated to market participants.

Historically, cross-border capacity allocation in European electricity markets has relied on the \gls{ATC} methodology. Under ATC, each transmission corridor is assigned a maximum transfer capacity value, the \gls{NTC}, which represents the maximum power that can be exchanged between two adjacent zones while respecting security constraints. These NTC values are determined by \glspl{TSO} based on worst-case assumptions about network conditions and are typically conservative to ensure N-1 security under all anticipated operating conditions~\cite{eu_cacm_2015}.

The fundamental limitation of ATC lies in its treatment of the physical network. By reducing the complex, meshed transmission grid to simple bilateral capacity limits, ATC cannot capture the physical reality that power flows distribute across multiple parallel paths according to Kirchhoff's laws. In meshed networks, a transaction between zones A and B will cause flows not only on the direct A-B corridor but also on alternative paths through intermediate zones---so-called loop flows. The ATC methodology must embed conservative margins to account for these effects, resulting in systematic underutilization of actual physical capacity.

\gls{FBMC} addresses this limitation by explicitly modeling the physical network through \gls{PTDF} matrices. PTDFs quantify how a unit injection at one node and withdrawal at another distributes across all network elements. By incorporating these factors into the market clearing algorithm, FBMC can identify feasible dispatch solutions that would be rejected under ATC's simplified constraints, while still respecting actual physical limits on each \gls{CNEC}.

The \gls{CWE} region pioneered FBMC implementation in May 2015, with subsequent analyses documenting welfare gains of 150--300 million EUR annually~\cite{acer_fbmc_2019}. Building on this experience, the Nordic electricity market---comprising Sweden, Norway, Finland, and Denmark---completed its transition to FBMC in October 2024 following an 18-month parallel operation period coordinated by the Nordic \gls{RCC}.

\section{Problem Definition}
\label{sec:problem}

The Swedish transmission system presents a compelling case study for analyzing FBMC effectiveness. Sweden's four bidding zones (SE1--SE4) exhibit a predominantly north-south radial topology, with the critical SE2--SE3 corridor representing the primary bottleneck constraining power transfer from hydroelectric-dominated northern regions to load-intensive southern consumption centers. This structural characteristic raises fundamental questions regarding the magnitude of efficiency gains achievable through FBMC in systems with limited internal loop flow opportunities.

\subsection{Research Questions}

The primary research question addressed by this thesis is:

\textbf{What is the quantitative difference in market efficiency between \gls{FBMC} and \gls{ATC} methods for the Swedish electricity market, and how does this difference vary with seasonal demand patterns and network congestion levels?}

This overarching question is decomposed into the following specific research objectives:

\begin{enumerate}
    \item \textbf{RQ1 (Efficiency Differential):} What is the magnitude of cost savings achieved by FBMC compared to ATC under representative operating conditions?

    \item \textbf{RQ2 (Topology Dependency):} How does the relative performance of FBMC versus ATC depend on network topology, specifically comparing radial internal transmission with meshed cross-border interconnections?

    \item \textbf{RQ3 (Seasonal Variation):} How do seasonal variations in demand patterns and congestion levels influence the distribution of FBMC benefits throughout the year?

    \item \textbf{RQ4 (Robustness):} How robust are the observed FBMC benefits to variations in exogenous parameters including neighboring zone prices and transmission capacities?
\end{enumerate}

\section{Purpose}
\label{sec:purpose}

The purpose of this thesis is twofold. First, from an academic perspective, it contributes to the understanding of capacity allocation mechanisms by providing quantitative evidence on FBMC performance in a system with distinct topological characteristics compared to the well-studied CWE region. Second, from a practical perspective, the findings inform policy discussions regarding Nordic market design and transmission investment planning.

The beneficiaries of this work include:
\begin{itemize}
    \item Nordic TSOs evaluating FBMC implementation outcomes
    \item Regulatory authorities assessing market efficiency
    \item Market participants developing trading strategies
    \item Academic researchers studying European electricity market integration
\end{itemize}

\section{Goals}
\label{sec:goals}

The concrete goals of this degree project are:

\begin{enumerate}
    \item \textbf{Model Development:} Develop a validated zonal market model capable of simulating both FBMC and ATC capacity allocation for the Nordic region

    \item \textbf{Quantitative Analysis:} Quantify the efficiency differential between FBMC and ATC across multiple time periods and scenarios

    \item \textbf{Mechanism Identification:} Identify the physical and economic mechanisms driving observed differences

    \item \textbf{Policy Implications:} Derive actionable insights for transmission planning and market design
\end{enumerate}

\section{Research Methodology}
\label{sec:methodology_intro}

This thesis employs quantitative simulation methodology combined with counterfactual analysis. A zonal market model is developed and validated against historical data, then used to simulate market outcomes under both FBMC and ATC regimes while holding all other factors constant. The difference in outcomes represents the causal effect of the capacity allocation methodology.

The choice of simulation methodology is motivated by several factors:
\begin{itemize}
    \item Controlled comparison: Real-world data cannot provide clean counterfactuals since only one methodology is operational at any time
    \item Parameter variation: Simulation enables systematic sensitivity analysis
    \item Reproducibility: Open-source tools and public data ensure results can be verified
\end{itemize}

Alternative methodologies considered but rejected include econometric analysis of before/after FBMC implementation (confounded by other time-varying factors) and agent-based modeling (added complexity without proportional insight for this research question).

\section{Delimitations}
\label{sec:delimitations}

This thesis is explicitly bounded by the following delimitations:

\begin{itemize}
    \item \textbf{Geographic scope:} Analysis focuses on the Swedish bidding zones with neighboring areas modeled as price-taking zones
    \item \textbf{Temporal scope:} Representative weeks from 2024 are analyzed; continuous annual simulation is not performed
    \item \textbf{Market scope:} Only the day-ahead market is modeled; intraday and balancing markets are excluded
    \item \textbf{Technical scope:} DC power flow approximation is used; AC power flow effects are not captured
    \item \textbf{Strategic behavior:} Market participants are assumed to bid at marginal cost; strategic bidding is not modeled
\end{itemize}

\section{Thesis Structure}
\label{sec:structure}

The remainder of this thesis is organized as follows:

Chapter~\ref{ch:background} provides theoretical background on electricity market design, FBMC methodology, and reviews related work on capacity allocation analysis.

Chapter~\ref{ch:methods} describes the research methodology, including the mathematical formulation, data sources, and experimental design.

Chapter~\ref{ch:implementation} details the model implementation, including PTDF calculation and optimization setup.

Chapter~\ref{ch:results} presents the results of base case, sensitivity, and multi-period analyses.

Chapter~\ref{ch:discussion} interprets the results and discusses their implications.

Chapter~\ref{ch:conclusions} summarizes conclusions, limitations, and directions for future work.


%%%%%%%%%%%%%%%%%%%%%%%%%%%%%%%%%%%%%%%%%%%%%%%%%%%%%%%%%%%%%%%%%%
%% CHAPTER 2: BACKGROUND
%%%%%%%%%%%%%%%%%%%%%%%%%%%%%%%%%%%%%%%%%%%%%%%%%%%%%%%%%%%%%%%%%%
\cleardoublepage
\chapter{Background}
\label{ch:background}

This chapter provides the theoretical foundation for understanding capacity allocation in electricity markets. Section~\ref{sec:electricity_markets} introduces electricity market fundamentals. Section~\ref{sec:atc_theory} describes the ATC methodology. Section~\ref{sec:fbmc_theory} presents FBMC theory. Section~\ref{sec:nordic_market} describes the Nordic electricity market. Section~\ref{sec:related_work} reviews related academic work.

\section{Electricity Market Fundamentals}
\label{sec:electricity_markets}

Modern electricity markets operate through a sequence of trading platforms with different time horizons. The day-ahead market, operated by power exchanges such as Nord Pool in the Nordic region, determines hourly prices and quantities for the following day through a uniform-price auction. Market participants submit bids specifying the quantity they wish to buy or sell at various price levels, and the market operator determines the market-clearing price where aggregate supply equals aggregate demand.

In zonal pricing systems, the market is divided into bidding zones within which a single price applies. When transmission capacity between zones is sufficient to accommodate all desired trades, prices converge across zones. When transmission constraints bind, zones decouple and different prices emerge, reflecting the marginal cost of congestion.

\section{Available Transfer Capacity Method}
\label{sec:atc_theory}

Under the ATC methodology, each interconnector between bidding zones is assigned a \gls{NTC} value representing the maximum commercial exchange permitted. The market clearing problem can be formulated as:

\begin{equation}
\min_{g, f} \sum_{n} C_n \cdot g_n
\end{equation}

subject to:
\begin{align}
\sum_{g \in G_n} g - D_n + \sum_{l: to(l)=n} f_l - \sum_{l: from(l)=n} f_l &= 0 \quad \forall n \\
|f_l| &\leq NTC_l \quad \forall l \\
g_n^{min} &\leq g_n \leq g_n^{max} \quad \forall n
\end{align}

where $g_n$ is generation at node $n$, $C_n$ is marginal cost, $D_n$ is demand, and $f_l$ is flow on corridor $l$.

The key limitation is that NTC values must be set conservatively to ensure security under all anticipated flow patterns, since the ATC formulation cannot distinguish between different flow distributions that result in the same net exchange.

\section{Flow-Based Market Coupling}
\label{sec:fbmc_theory}

FBMC replaces the simplified NTC constraints with explicit representation of power flow physics through \gls{PTDF} matrices. The PTDF coefficient $PTDF_{l,n}$ quantifies the fraction of a unit injection at node $n$ (with corresponding withdrawal at the slack bus) that flows through line $l$.

For a DC power flow approximation, PTDFs are calculated as:
\begin{equation}
PTDF = B_f \cdot B^{-1}
\end{equation}

where $B_f$ is the branch susceptance matrix and $B$ is the nodal susceptance matrix.

The FBMC market clearing problem becomes:
\begin{equation}
\min_{g, p} \sum_{n} C_n \cdot g_n
\end{equation}

subject to:
\begin{align}
g_n - D_n &= p_n \quad \forall n \\
\left| \sum_{n} PTDF_{l,n} \cdot p_n \right| &\leq F_l^{max} \quad \forall l \in CNEC \\
g_n^{min} &\leq g_n \leq g_n^{max} \quad \forall n
\end{align}

where $p_n$ is net position (injection minus withdrawal) at node $n$ and $F_l^{max}$ is the thermal limit of line $l$.

The critical advantage is that FBMC constraints reflect actual physical flows, allowing the market to exploit parallel paths and loop flows that ATC cannot represent.

\section{The Nordic Electricity Market}
\label{sec:nordic_market}

The Nordic electricity market, operated by Nord Pool, comprises multiple bidding zones across Sweden (SE1--SE4), Norway (NO1--NO5), Finland (FI), and Denmark (DK1--DK2). The market is characterized by:

\begin{itemize}
    \item Large hydropower resources in Norway and northern Sweden
    \item Nuclear generation in Sweden and Finland
    \item Growing wind power capacity across all countries
    \item Significant cross-border interconnection capacity
    \item Seasonal demand patterns with winter peaks
\end{itemize}

Sweden's internal transmission system follows a predominantly north-south orientation, transferring hydroelectric generation from SE1 and SE2 to consumption centers in SE3 and SE4. The SE2--SE3 corridor, with an NTC of approximately 7,300 MW, represents the critical bottleneck in this system.

\section{Related Work}
\label{sec:related_work}

The academic literature on FBMC has grown substantially since the CWE implementation. Weinhold and Mieth~\cite{weinhold_pomato_2021} developed POMATO, an open-source tool for flow-based market analysis that has become a reference implementation. Their work demonstrated the importance of \gls{GSK} assumptions in determining FBMC outcomes.

Schönheit et al.~\cite{schonheit_gsk_2020} systematically analyzed how different GSK strategies affect the FBMC domain in CWE, finding that the choice of GSK methodology can significantly impact cross-border capacity availability.

Van den Bergh et al.~\cite{van_den_bergh_dc_2014} examined the validity of DC power flow assumptions in market models, providing theoretical justification for the approximations used in operational FBMC implementations.

The literature specific to Nordic FBMC remains limited given the recent implementation. This thesis contributes by providing early quantitative evidence on FBMC performance in the Nordic context.

\section{Summary}
\label{sec:background_summary}

This chapter has established the theoretical foundation for comparing FBMC and ATC capacity allocation. Key insights include:

\begin{itemize}
    \item ATC simplifies the network to bilateral capacity limits, requiring conservative margins
    \item FBMC explicitly models power flow physics through PTDF matrices
    \item The Nordic market's north-south topology creates specific patterns of congestion
    \item Existing literature demonstrates FBMC benefits in CWE but Nordic-specific analysis is lacking
\end{itemize}


%%%%%%%%%%%%%%%%%%%%%%%%%%%%%%%%%%%%%%%%%%%%%%%%%%%%%%%%%%%%%%%%%%
%% CHAPTER 3: METHODS
%%%%%%%%%%%%%%%%%%%%%%%%%%%%%%%%%%%%%%%%%%%%%%%%%%%%%%%%%%%%%%%%%%
\cleardoublepage
\chapter{Methods}
\label{ch:methods}

This chapter describes the research methodology employed in this thesis. Section~\ref{sec:research_process} outlines the overall research process. Section~\ref{sec:model_formulation} presents the mathematical model formulation. Section~\ref{sec:data_collection} describes data sources. Section~\ref{sec:experimental_design} details the experimental design. Section~\ref{sec:validity_reliability} discusses validity and reliability considerations.

\section{Research Process}
\label{sec:research_process}

The research follows a quantitative simulation methodology with the following phases:

\begin{enumerate}
    \item \textbf{Model Development:} Formulation of zonal market clearing problem for both FBMC and ATC
    \item \textbf{Data Collection:} Acquisition of network, generation, demand, and price data from official sources
    \item \textbf{Implementation:} Python-based implementation using scipy optimization
    \item \textbf{Validation:} Comparison of model outputs against historical market outcomes
    \item \textbf{Analysis:} Systematic comparison of FBMC and ATC across scenarios
    \item \textbf{Interpretation:} Synthesis of findings and derivation of implications
\end{enumerate}

\section{Mathematical Model Formulation}
\label{sec:model_formulation}

\subsection{Network Representation}

The model employs a 13-node zonal representation comprising:
\begin{itemize}
    \item Four Swedish bidding zones: SE1, SE2, SE3, SE4
    \item Nine neighboring zones: NO1, NO3, NO4, FI, DK1, DK2, DE, PL, LT
\end{itemize}

The network includes 13 transmission corridors: three internal Swedish (SE1--SE2, SE2--SE3, SE3--SE4) and ten cross-border interconnectors.

\subsection{Objective Function}

Both FBMC and ATC formulations minimize total system cost:
\begin{equation}
\min \sum_{n \in N} \sum_{t \in T} C_n \cdot g_{n,t}
\end{equation}

where $N$ is the set of nodes, $T$ is the set of time periods, $C_n$ is marginal cost at node $n$, and $g_{n,t}$ is generation.

\subsection{Constraints}

\textbf{Power Balance:}
\begin{equation}
\sum_{u \in U_n} g_{u,t} - D_{n,t} = p_{n,t} \quad \forall n, t
\end{equation}

\textbf{Generation Limits:}
\begin{equation}
0 \leq g_{u,t} \leq G_u^{max} \quad \forall u, t
\end{equation}

\textbf{FBMC Flow Constraints:}
\begin{equation}
-F_l^{max} \leq \sum_{n} PTDF_{l,n} \cdot p_{n,t} \leq F_l^{max} \quad \forall l, t
\end{equation}

\textbf{ATC Flow Constraints:}
\begin{equation}
-NTC_l \leq f_{l,t} \leq NTC_l \quad \forall l, t
\end{equation}

\subsection{PTDF Calculation}

PTDFs are computed using the DC power flow approximation:
\begin{equation}
PTDF_{l,n} = \frac{1}{x_l} \left( \psi_{from(l),n} - \psi_{to(l),n} \right)
\end{equation}

where $x_l$ is line reactance and $\psi$ is the bus injection shift factor matrix derived from the network admittance matrix with SE3 as slack bus.

\section{Data Collection}
\label{sec:data_collection}

All data derives from authoritative public sources to ensure reproducibility:

\textbf{ENTSO-E Transparency Platform~\cite{entsoe_transparency_2024}:}
\begin{itemize}
    \item Day-ahead prices (document A44): Hourly prices for all bidding zones
    \item Actual total load (document A65): Hourly demand by zone
    \item Generation per type (document A75): Hourly generation by technology
\end{itemize}

\textbf{Svenska kraftnät~\cite{svk_kraftbalansen_2024}:}
\begin{itemize}
    \item Installed generation capacity by zone and technology
    \item Internal NTC values: SE1--SE2 (3,300 MW), SE2--SE3 (7,300 MW), SE3--SE4 (5,300 MW)
\end{itemize}

\textbf{PyPSA-Eur~\cite{horsch_pypsa_2018}:}
\begin{itemize}
    \item Transmission line electrical parameters (reactances) for PTDF calculation
\end{itemize}

\subsection{Marginal Cost Assumptions}

Generation marginal costs are assigned by technology:
\begin{itemize}
    \item Hydro: 0.5 EUR/MWh (water value approximation)
    \item Nuclear: 10 EUR/MWh
    \item Wind/Solar: 0 EUR/MWh
    \item CHP: 35--40 EUR/MWh
    \item Gas: 50 EUR/MWh
    \item Coal: 80 EUR/MWh
\end{itemize}

\section{Experimental Design}
\label{sec:experimental_design}

\subsection{Base Case Analysis}

The base case simulates one representative week (168 hours) from December 2024 under both FBMC and ATC regimes. December represents winter conditions with high demand and significant north-south price spreads.

\subsection{Multi-Period Analysis}

To capture seasonal variation, additional analyses are conducted for:
\begin{itemize}
    \item January 2024: Winter peak demand
    \item July 2024: Summer minimum demand
\end{itemize}

\subsection{Sensitivity Analysis}

Five sensitivity scenarios test robustness:
\begin{itemize}
    \item \textbf{Scenario A:} Continental prices +30\%
    \item \textbf{Scenario B:} Norwegian prices -30\%
    \item \textbf{Scenario C:} Price convergence (uniform 60 EUR/MWh)
    \item \textbf{Scenario D:} SE2--SE3 capacity -30\%
    \item \textbf{Scenario E:} Cross-border capacity +20\%
\end{itemize}

\section{Validity and Reliability}
\label{sec:validity_reliability}

\subsection{Internal Validity}

Internal validity is ensured through:
\begin{itemize}
    \item Identical input data for FBMC and ATC comparisons
    \item Systematic parameter variation in sensitivity analysis
    \item Verification of optimization convergence
\end{itemize}

\subsection{External Validity}

Limitations to external validity include:
\begin{itemize}
    \item Zonal aggregation may not capture all network effects
    \item Static NTC values (actual values vary hourly)
    \item Marginal cost assumptions may differ from actual bids
\end{itemize}

\subsection{Reliability}

Reliability is supported by:
\begin{itemize}
    \item Use of established optimization solver (HiGHS)
    \item Public data sources enabling replication
    \item Documentation of all assumptions and parameters
\end{itemize}


%%%%%%%%%%%%%%%%%%%%%%%%%%%%%%%%%%%%%%%%%%%%%%%%%%%%%%%%%%%%%%%%%%
%% CHAPTER 4: IMPLEMENTATION
%%%%%%%%%%%%%%%%%%%%%%%%%%%%%%%%%%%%%%%%%%%%%%%%%%%%%%%%%%%%%%%%%%
\cleardoublepage
\chapter{Implementation}
\label{ch:implementation}

This chapter describes the technical implementation of the market clearing model. Section~\ref{sec:software_tools} describes the software environment. Section~\ref{sec:ptdf_implementation} details PTDF matrix calculation. Section~\ref{sec:optimization_setup} presents the optimization implementation.

\section{Software Tools}
\label{sec:software_tools}

The model is implemented in Python 3.11 using the following libraries:
\begin{itemize}
    \item \texttt{scipy.optimize.linprog}: Linear programming solver interface
    \item \texttt{numpy}: Numerical array operations
    \item \texttt{pandas}: Data manipulation and analysis
    \item \texttt{HiGHS}: High-performance LP solver (via scipy)
\end{itemize}

The implementation follows the methodological framework documented in the POMATO tool~\cite{weinhold_pomato_2021}, though using a custom Python implementation rather than POMATO's Julia backend.

\section{PTDF Matrix Calculation}
\label{sec:ptdf_implementation}

The PTDF calculation proceeds in the following steps:

\begin{enumerate}
    \item Construct nodal admittance matrix $Y$ from line reactances
    \item Extract susceptance matrix $B$ (imaginary part of $Y$)
    \item Remove slack bus row/column to form reduced matrix $B'$
    \item Compute bus injection shift factors: $\Psi = (B')^{-1}$
    \item For each line $l$: $PTDF_{l,n} = \frac{1}{x_l}(\Psi_{from(l),n} - \Psi_{to(l),n})$
\end{enumerate}

SE3 is designated as the slack bus, representing the primary load center and price reference point.

Figure~\ref{fig:network_topology} shows the 13-node network topology used in the model.

\begin{figure}[!ht]
  \begin{center}
    \includegraphics[width=0.8\textwidth]{figures/network_topology_extended.png}
  \end{center}
  \caption{Network topology of the 13-node Nordic market model showing Swedish bidding zones (SE1--SE4) and neighboring market areas}
  \label{fig:network_topology}
\end{figure}

Figure~\ref{fig:ptdf_heatmap} presents the calculated PTDF matrix as a heatmap, showing how injections at each node affect flows on each transmission corridor.

\begin{figure}[!ht]
  \begin{center}
    \includegraphics[width=0.9\textwidth]{figures/ptdf_heatmap_v3.png}
  \end{center}
  \caption{PTDF matrix heatmap showing the sensitivity of line flows to nodal injections}
  \label{fig:ptdf_heatmap}
\end{figure}

\section{Optimization Setup}
\label{sec:optimization_setup}

The market clearing problem is formulated in standard LP form:
\begin{equation}
\min_x c^T x \quad \text{s.t.} \quad A_{eq} x = b_{eq}, \quad A_{ub} x \leq b_{ub}, \quad lb \leq x \leq ub
\end{equation}

Decision variables $x$ include generation quantities for each unit and time period. The constraint matrices encode power balance (equality) and flow limits (inequality). The HiGHS solver is invoked through scipy's interface with default tolerances.

Typical solution times are under 10 seconds for a 168-hour simulation on standard hardware.


%%%%%%%%%%%%%%%%%%%%%%%%%%%%%%%%%%%%%%%%%%%%%%%%%%%%%%%%%%%%%%%%%%
%% CHAPTER 5: RESULTS AND ANALYSIS
%%%%%%%%%%%%%%%%%%%%%%%%%%%%%%%%%%%%%%%%%%%%%%%%%%%%%%%%%%%%%%%%%%
\cleardoublepage
\chapter{Results and Analysis}
\label{ch:results}

This chapter presents the results of the FBMC versus ATC comparison. Section~\ref{sec:base_case_results} presents base case findings. Section~\ref{sec:sensitivity_results} reports sensitivity analysis. Section~\ref{sec:seasonal_results} examines seasonal variation. Section~\ref{sec:mechanism_analysis} analyzes the underlying mechanisms.

\section{Base Case Results}
\label{sec:base_case_results}

Table~\ref{tab:base_case} summarizes the base case results for December 2024.

\begin{table}[ht]
\centering
\caption{Base Case Results: December 2024}
\label{tab:base_case}
\begin{tabular}{lrr}
\hline
\textbf{Metric} & \textbf{FBMC} & \textbf{ATC} \\
\hline
Total System Cost (M EUR) & -71.1 & -66.7 \\
Weekly Savings (M EUR) & \multicolumn{2}{c}{4.37} \\
Percentage Savings & \multicolumn{2}{c}{6.5\%} \\
SE2--SE3 Utilization & 88\% & 99\% \\
Average SE3 Price (EUR/MWh) & 42.3 & 48.7 \\
\hline
\end{tabular}
\end{table}

The negative costs reflect the model's treatment of demand as negative generation with associated welfare. FBMC achieves 6.5\% cost reduction compared to ATC, equivalent to 4.37 million EUR weekly savings.

The key observation is that FBMC reduces utilization on the SE2--SE3 corridor from 99\% to 88\% while achieving better overall outcomes. This occurs because FBMC can route some power through the NO4--NO3--NO1 loop path, utilizing cross-border capacity that ATC treats as independent.

Figure~\ref{fig:price_timeseries} shows the hourly price evolution under both methodologies for the base case period.

\begin{figure}[!ht]
  \begin{center}
    \includegraphics[width=0.95\textwidth]{figures/price_timeseries_v2.png}
  \end{center}
  \caption{Hourly price time series for Swedish bidding zones under FBMC and ATC}
  \label{fig:price_timeseries}
\end{figure}

Figure~\ref{fig:line_utilization} compares line utilization between FBMC and ATC across all transmission corridors.

\begin{figure}[!ht]
  \begin{center}
    \includegraphics[width=0.9\textwidth]{figures/line_utilization_comparison_v2.png}
  \end{center}
  \caption{Comparison of transmission line utilization between FBMC and ATC}
  \label{fig:line_utilization}
\end{figure}

\section{Sensitivity Analysis Results}
\label{sec:sensitivity_results}

Table~\ref{tab:sensitivity} presents sensitivity analysis results across five scenarios.

\begin{table}[ht]
\centering
\caption{Sensitivity Analysis Results}
\label{tab:sensitivity}
\begin{tabular}{llr}
\hline
\textbf{Scenario} & \textbf{Description} & \textbf{FBMC Savings} \\
\hline
Base & December 2024 conditions & 6.5\% \\
A & Continental prices +30\% & 4.9\% \\
B & Norwegian prices -30\% & 4.6\% \\
C & Price convergence (60 EUR/MWh) & 0.0\% \\
D & SE2--SE3 capacity -30\% & 16.1\% \\
E & Cross-border capacity +20\% & 6.7\% \\
\hline
\end{tabular}
\end{table}

Key findings from sensitivity analysis:

\begin{itemize}
    \item \textbf{Scenario C} confirms the arbitrage mechanism: when prices converge, FBMC provides zero benefit as there is no value in redirecting flows
    \item \textbf{Scenario D} shows FBMC benefits increase dramatically (16.1\%) under constrained conditions
    \item \textbf{Scenarios A and B} demonstrate robustness: FBMC maintains positive benefits across price variations
\end{itemize}

Figure~\ref{fig:sensitivity} provides a visual comparison of FBMC performance across all sensitivity scenarios.

\begin{figure}[!ht]
  \begin{center}
    \includegraphics[width=0.9\textwidth]{figures/sensitivity_comparison.png}
  \end{center}
  \caption{FBMC cost savings across sensitivity scenarios}
  \label{fig:sensitivity}
\end{figure}

\section{Seasonal Variation Results}
\label{sec:seasonal_results}

Table~\ref{tab:seasonal} compares results across seasons.

\begin{table}[ht]
\centering
\caption{Seasonal Variation in FBMC Benefits}
\label{tab:seasonal}
\begin{tabular}{lrrr}
\hline
\textbf{Period} & \textbf{FBMC Savings} & \textbf{SE2--SE3 Util.} & \textbf{Avg. Spread} \\
\hline
January 2024 & 12.6\% & 92\% & 18.4 EUR/MWh \\
December 2024 & 6.5\% & 88\% & 12.1 EUR/MWh \\
July 2024 & 0.25\% & 53\% & 2.3 EUR/MWh \\
\hline
\end{tabular}
\end{table}

The results reveal a 50-fold variation in FBMC benefits between winter (12.6\%) and summer (0.25\%). This dramatic seasonal pattern is driven by:

\begin{itemize}
    \item Winter: High demand, constrained transmission, large price spreads
    \item Summer: Low demand, uncongested transmission, minimal price spreads
\end{itemize}

Figure~\ref{fig:multi_period} shows the detailed comparison across all analyzed time periods.

\begin{figure}[!ht]
  \begin{center}
    \includegraphics[width=0.95\textwidth]{figures/multi_period_comparison.png}
  \end{center}
  \caption{Multi-period comparison of FBMC versus ATC performance}
  \label{fig:multi_period}
\end{figure}

\section{Mechanism Analysis}
\label{sec:mechanism_analysis}

Analysis of the underlying mechanisms confirms the theoretical expectations:

\textbf{Topology Effect:} FBMC provides minimal benefit for Sweden's internal radial corridors (SE1--SE2--SE3--SE4) where power flow paths are uniquely determined. Benefits emerge only when cross-border connections create loops, specifically the SE2--NO4--NO3--NO1--SE3 path that allows alternative routing.

\textbf{Congestion Correlation:} Figure~\ref{fig:congestion_benefit} shows the strong positive correlation between SE2--SE3 utilization and FBMC benefit magnitude.

\begin{figure}[!ht]
  \begin{center}
    \includegraphics[width=0.8\textwidth]{figures/congestion_benefit.png}
  \end{center}
  \caption{Relationship between congestion level and FBMC benefit}
  \label{fig:congestion_benefit}
\end{figure}

\textbf{Arbitrage Mechanism:} The complete elimination of FBMC benefits under price convergence (Scenario C) confirms that the mechanism operates through enhanced arbitrage facilitation. FBMC creates value by enabling trades that ATC's conservative constraints would block.

Figure~\ref{fig:cross_border} shows the cross-border flow patterns that enable FBMC optimization.

\begin{figure}[!ht]
  \begin{center}
    \includegraphics[width=0.85\textwidth]{figures/cross_border_flows.png}
  \end{center}
  \caption{Cross-border flow patterns under FBMC and ATC}
  \label{fig:cross_border}
\end{figure}


%%%%%%%%%%%%%%%%%%%%%%%%%%%%%%%%%%%%%%%%%%%%%%%%%%%%%%%%%%%%%%%%%%
%% CHAPTER 6: DISCUSSION
%%%%%%%%%%%%%%%%%%%%%%%%%%%%%%%%%%%%%%%%%%%%%%%%%%%%%%%%%%%%%%%%%%
\cleardoublepage
\chapter{Discussion}
\label{ch:discussion}

This chapter interprets the results and discusses their broader implications.

\section{Interpretation of Results}

The findings support the hypotheses formulated in Chapter~\ref{ch:introduction}:

\begin{itemize}
    \item \textbf{H1 (Positive Welfare Effect):} Confirmed. FBMC achieves 6.5\% savings in winter conditions.
    \item \textbf{H2 (Topology Threshold):} Confirmed. Benefits require cross-border loop flows.
    \item \textbf{H3 (Congestion Amplification):} Confirmed. Benefits scale with congestion (16.1\% under constrained scenario).
    \item \textbf{H4 (Arbitrage Dependency):} Confirmed. Benefits disappear under price convergence.
\end{itemize}

\section{Comparison with Literature}

The 6.5\% winter savings observed in this study is higher than typical CWE estimates of 1--3\% welfare improvement. This may reflect:

\begin{itemize}
    \item More severe baseline congestion on SE2--SE3
    \item Different network topology creating larger optimization potential
    \item Modeling assumptions that may overstate differences
\end{itemize}

\section{Policy Implications}

Several policy implications emerge:

\begin{enumerate}
    \item \textbf{Transmission Planning:} FBMC benefits are concentrated during high-congestion periods, suggesting targeted transmission investments may be cost-effective

    \item \textbf{Market Monitoring:} Regulators should track FBMC performance metrics seasonally, not just annually

    \item \textbf{Bidding Zone Review:} The strong north-south congestion pattern supports ongoing discussions about Swedish bidding zone configuration
\end{enumerate}

\section{Limitations}

Several limitations should be acknowledged:

\begin{itemize}
    \item The zonal aggregation may not capture all relevant network effects
    \item Static NTC values may understate ATC flexibility
    \item Marginal cost assumptions may differ from actual market bids
    \item Strategic bidding effects are not modeled
    \item Only three time periods are analyzed
\end{itemize}


%%%%%%%%%%%%%%%%%%%%%%%%%%%%%%%%%%%%%%%%%%%%%%%%%%%%%%%%%%%%%%%%%%
%% CHAPTER 7: CONCLUSIONS AND FUTURE WORK
%%%%%%%%%%%%%%%%%%%%%%%%%%%%%%%%%%%%%%%%%%%%%%%%%%%%%%%%%%%%%%%%%%
\cleardoublepage
\chapter{Conclusions and Future Work}
\label{ch:conclusions}

\section{Conclusions}
\label{sec:conclusions}

This thesis has provided a quantitative assessment of Flow-Based Market Coupling versus Available Transfer Capacity for the Nordic electricity market, with focus on the Swedish transmission system.

The main conclusions are:

\begin{enumerate}
    \item \textbf{FBMC provides measurable efficiency gains over ATC}, with 6.5\% cost savings observed during winter conditions when the SE2--SE3 corridor is congested.

    \item \textbf{Benefits exhibit strong seasonal variation}, with a 50-fold difference between winter (12.6\%) and summer (0.25\%) driven by congestion patterns.

    \item \textbf{Network topology determines benefit availability}: FBMC provides value only when cross-border connections create loop flow opportunities; purely radial sections see no improvement.

    \item \textbf{The mechanism operates through arbitrage facilitation}: benefits disappear entirely when price spreads converge, confirming the economic driver.

    \item \textbf{Benefits scale with system stress}: under constrained transmission scenarios, FBMC benefits increase to 16.1\%, suggesting FBMC is most valuable precisely when efficient capacity allocation matters most.
\end{enumerate}

These findings have implications for Nordic transmission planning and market design, suggesting that FBMC implementation was justified and that its benefits should be monitored on a seasonal basis.

\section{Limitations}
\label{sec:limitations}

The analysis is subject to limitations including:

\begin{itemize}
    \item Zonal aggregation simplifying network representation
    \item Limited number of time periods analyzed
    \item Static parameter assumptions (NTC, costs)
    \item No modeling of strategic bidding behavior
    \item Custom implementation rather than production FBMC tools
\end{itemize}

\section{Future Work}
\label{sec:future_work}

Several directions for future work are identified:

\begin{enumerate}
    \item \textbf{Extended temporal analysis:} Full annual simulation to capture all seasonal patterns
    \item \textbf{Validation against actual market data:} Comparison with Nordic RCC parallel run results
    \item \textbf{Higher resolution modeling:} Transition from zonal to nodal representation
    \item \textbf{Integration with N-1 security:} Explicit modeling of contingency constraints
    \item \textbf{Welfare distribution analysis:} Examination of how FBMC benefits are distributed across market participants
\end{enumerate}

\section{Reflections on Sustainability and Ethics}
\label{sec:reflections}

This thesis contributes to the United Nations Sustainable Development Goals, particularly:

\textbf{SDG 7 (Affordable and Clean Energy):} Efficient capacity allocation enables better integration of renewable energy resources, particularly hydropower from northern Sweden, reducing the need for fossil fuel generation in southern regions.

\textbf{SDG 9 (Industry, Innovation and Infrastructure):} The analysis supports informed infrastructure investment decisions by quantifying the value of transmission capacity under different allocation methodologies.

\textbf{SDG 13 (Climate Action):} Improved market efficiency reduces overall system costs and enables higher renewable penetration, contributing to decarbonization objectives.

From an ethical perspective, this research relies exclusively on public data sources and open-source tools, ensuring transparency and reproducibility. The findings are presented objectively, acknowledging limitations and avoiding overstatement of conclusions.


%%%%%%%%%%%%%%%%%%%%%%%%%%%%%%%%%%%%%%%%%%%%%%%%%%%%%%%%%%%%%%%%%%
%% BIBLIOGRAPHY
%%%%%%%%%%%%%%%%%%%%%%%%%%%%%%%%%%%%%%%%%%%%%%%%%%%%%%%%%%%%%%%%%%
\cleardoublepage

\iftoggle{biblatex}{
    \printbibliography[heading=bibintoc, title=References]
}{
    \renewcommand{\bibname}{References}
    \addcontentsline{toc}{chapter}{References}
    \bibliography{references}
}


%%%%%%%%%%%%%%%%%%%%%%%%%%%%%%%%%%%%%%%%%%%%%%%%%%%%%%%%%%%%%%%%%%
%% APPENDICES
%%%%%%%%%%%%%%%%%%%%%%%%%%%%%%%%%%%%%%%%%%%%%%%%%%%%%%%%%%%%%%%%%%
\cleardoublepage
\appendix
\renewcommand{\chaptermark}[1]{\markboth{Appendix \thechapter\relax:\thinspace\relax#1}{}}

\chapter{PTDF Matrix for Swedish Network}
\label{app:ptdf}

Table~\ref{tab:ptdf_excerpt} shows an excerpt of the calculated PTDF matrix for key corridors.

\begin{table}[ht]
\centering
\caption{PTDF Matrix Excerpt (Selected Corridors)}
\label{tab:ptdf_excerpt}
\begin{tabular}{lrrrr}
\hline
\textbf{Corridor} & \textbf{SE1} & \textbf{SE2} & \textbf{SE3} & \textbf{SE4} \\
\hline
SE1--SE2 & 1.000 & 0.000 & 0.000 & 0.000 \\
SE2--SE3 & 0.714 & 0.714 & 0.000 & 0.000 \\
SE3--SE4 & 0.286 & 0.286 & 0.286 & 0.000 \\
SE2--NO4 & 0.286 & 0.286 & 0.000 & 0.000 \\
\hline
\end{tabular}
\end{table}

The non-trivial PTDF values for SE2--NO4 (0.286) indicate the presence of loop flow paths that enable FBMC optimization.

\chapter{Data Sources and Parameters}
\label{app:data}

Complete parameter values and data sources are documented in this appendix for reproducibility.

\section{Generation Capacity by Zone}

\begin{table}[ht]
\centering
\caption{Installed Generation Capacity (MW)}
\label{tab:capacity}
\begin{tabular}{lrrrr}
\hline
\textbf{Technology} & \textbf{SE1} & \textbf{SE2} & \textbf{SE3} & \textbf{SE4} \\
\hline
Hydro & 5,200 & 8,100 & 1,800 & 200 \\
Nuclear & 0 & 0 & 6,900 & 0 \\
Wind & 2,100 & 4,800 & 3,200 & 1,600 \\
CHP & 300 & 400 & 2,100 & 1,200 \\
\hline
\end{tabular}
\end{table}

Source: Svenska kraftnät Kraftbalansen 2024~\cite{svk_kraftbalansen_2024}

\label{pg:lastPageofMainmatter}

\clearpage
\section*{For DIVA}
\divainfo{pg:lastPageofPreface}{pg:lastPageofMainmatter}

\end{document}
